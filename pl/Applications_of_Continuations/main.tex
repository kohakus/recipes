\documentclass[12pt]{article}
\usepackage{array}
\usepackage{amsmath}
\usepackage{amssymb}
\usepackage{xfrac}
\usepackage{ntheorem}
\usepackage{algorithm}
\usepackage{algorithmic}
\usepackage{caption}
\usepackage{fontspec}
\usepackage{graphicx}
\usepackage{indentfirst}
\usepackage{enumitem}
\usepackage{minted}
\usepackage{mathtools}
\usepackage{pifont}
\usepackage{setspace}
\usepackage{subfigure}
\usepackage{tikz}
\usepackage{url}
\usepackage{tcolorbox}
\usepackage{xcolor}
\usepackage{xeCJK}

\usepackage[colorlinks=true]{hyperref}
\usepackage[margin=0.55in]{geometry}

% background color for minted
\definecolor{bg}{rgb}{0.95,0.95,0.95}

% CJK font
\setCJKmainfont{Source Han Serif CN}

% indent value
\setlength{\parindent}{2em}

% line spacing
\linespread{1.2}
\renewcommand{\thesection}{\Roman{section}}
\renewcommand{\thesubsection}{\thesection-\arabic{subsection}}

\title{Applications of Continuations}
\author{Yiteng Zhang}

\begin{document}
\maketitle

\noindent{}\textbf{注:}本文译自Daniel P. Friedman的讲义文章\textsl{Applications of Continuations}(延续的应用)。
为了表达上的准确性,仍保留相关术语在原文中的描述。附录A未被包含,可直接参阅原文。
~\\

\noindent{}\textbf{文章的构成}:这篇讲义分为三个部分。第一部分我们主要通过例子来定义一些关键的术语。这些术语是
常规过程(conventional procedures),逃逸过程(escape procedures) 和延续(continuations)。\texttt{lambda\^{}}和\texttt{call/cc}
的原始概念会被展示。在第二部分中,我们会构建一些简单应用的例子,特别是一个简单的类-LISP的\texttt{BREAK}过程,和一个
无限循环但可从中跳出的\texttt{CYCLE}过程。在这一节中我们还指出,如果所有的过程都是以延续传递风格(continuation-passing-style)
写成的,那么\texttt{call/cc}并不是严格必须的。即使是 LISP 的 \texttt{PROG} 这种不受欢迎的 GO TO 风格的的编程,在正确
使用 \texttt{call/cc} 的情况下也显得很合理。在最后一节里,我们为同步进程开发了一个 \texttt{DISPATCHER}。最后我们还将两个
附录包含了进来。在附录A中,我们展示了一个用于一个包含\texttt{call/cc}和\texttt{lambda\^{}}的Scheme子集的完全柯里化版本的
元循环(meta-circular)解释器。在附录B中,我们包含了测试\texttt{DISPATCHER}所需的剩余代码。

\section{关键术语与定义}
\subsection{Scheme的特性}
\indent{}Scheme 编程语言是一种带有头等过程(first-class procedures)的传值调用(call-by-value)词法作用域LISP方言。
正确的尾递归(tail-recursive)是强加于其任何实现上的一个条件。简单来说,这意味着在不必要时不会有调用栈的增长。特别是,
以递归形式编写的简单循环的开销会最小化。Scheme语言具有一个函数式的子集,但其完整的语言集合支持一些命令式(imperative)的
概念,比如输入/输出(input/output),给词法变量(lexical variables)赋值,以及一级延续(first-class continuations)。

\subsection{逃逸过程(escape procedure)的描述}
\noindent{}让我们考虑下面的简单表达式:
\begin{minted}[mathescape=true,
               %linenos,
               numbersep=5pt,
               autogobble,
               frame=lines,
               fontsize=\small,
               bgcolor=bg,
               framesep=2mm]{Racket}
    (* (/ 24 (f 0)) 3)
\end{minted}
\noindent{}一些可能的结果:
\begin{enumerate}
    \item \texttt{(f 0)} 是 4,因而结果是18;
    \item \texttt{f}在0处未定义,引发一条错误信息,并使得计算过程被抛弃;
    \item \texttt{f}在0处未定义,导致无穷循环。比如说:
    \begin{minted}[mathescape=true,
               %linenos,
               numbersep=5pt,
               autogobble,
               frame=lines,
               fontsize=\small,
               bgcolor=bg,
               framesep=2mm]{Racket}
        (define f
          (lambda (n)
            (if (zero? n) (f n) n)))
    \end{minted}
    \item \texttt{(f 0)}是4,但是\texttt{f}是一个逃逸过程,因而结果是4。
\end{enumerate}

\subsection{用于表征逃逸过程的符号}
\indent{}我们引入一个记号来表征逃逸过程:如果\texttt{f}是一个过程,那么与它对应的逃逸过程是\texttt{f\^{}}。\texttt{f\^{}}
做的事情与\texttt{f}是一样的,不过\texttt{(f\^{} ...)}会作为结果逃逸出来,并随之被输出。有时这被称
为``逃逸到顶层''或``读取-求值-输出循环''(read-eval-print loop)。

\indent{}最简单的逃逸过程是与恒等函数\texttt{I}对应的过程。如果\texttt{f\^{}}$=$\texttt{I\^{}},那么之前的结果会是0,不会有
除法运算发生。一个更为强大的逃逸函数是\texttt{+\^{}},让我们考虑:
\begin{minted}[mathescape=true,
               %linenos,
               numbersep=5pt,
               autogobble,
               frame=lines,
               fontsize=\small,
               bgcolor=bg,
               framesep=2mm]{Racket}
    (* 3 (+^ 4 5))
\end{minted}
其结果是9。注意这和\texttt{(* 3 (I\^{} (+ 4 5)))}的结果相同。

\subsection{调用逃逸过程即是替换调用栈}
\indent{}对于上面的表达式,当在调用\texttt{(+\^{} 4 5)}时,调用栈的内容是<\texttt{3}, \texttt{*}>。调用逃逸过程
的作用是使调用栈当前的内容被放弃,只处理\texttt{+\^{}}和其参数。尽管\texttt{(I\^{} (+ 4 5))} 和 \texttt{(+\^{} 4 5)}
的结果相同,它们在具体操作上却并不相同。\texttt{(I\^{} (+ 4 5))}的调用栈会有更多的增长。这一点对调用栈增长的理解会
在后面有关更为复杂的逃逸过程的讨论中起到重要作用。

\subsection{\texttt{lambda\^{}}:一般逃逸过程的创建器}
\indent{}给定一个任意的 lambda 表达式,我们可以构建出与之对应的逃逸过程。也就是,从表达式\texttt{(lambda (x) ...)}
得到\texttt{(lambda\^{} (x) ...)}。目前我们先假设能够进行这种构建,至于具体如何从\texttt{lambda}过程构建\texttt{lambda\^{}},会
在后面进行阐述。

\subsection{\texttt{call/cc}:作为逃逸过程创建器}
\indent{}Scheme内置了用于创建逃逸过程的特性,并提供给用户使用。这些逃逸过程对应于运行时的调用栈,而运行时调用栈对应于延续语义中
的延续,因而这些特别的逃逸过程也常被称为延续或延续对象。
\begin{minted}[mathescape=true,
               %linenos,
               numbersep=5pt,
               autogobble,
               frame=lines,
               fontsize=\small,
               bgcolor=bg,
               framesep=2mm]{Racket}
    (call-with-current-continuation e)
\end{minted}
\noindent{}这行代码的效果是调用\texttt{(e k\^{})}。其中,\texttt{k\^{}}是对应于表达式
实例\texttt{(call-with-current-continuation e)}的 逃逸过程。这可以被简写成\texttt{(call/cc e)}。

\subsection{描述由\texttt{call/cc}创建的逃逸过程}
\noindent{}比如,如果我们有这样的表达式:
\begin{minted}[mathescape=true,
               %linenos,
               numbersep=5pt,
               autogobble,
               frame=lines,
               fontsize=\small,
               bgcolor=bg,
               framesep=2mm]{Racket}
    (+ 3 (call/cc (lambda (k^) ...)))
\end{minted}
\noindent{}那么\texttt{k\^{}}可以通过用一个新的变量(比如\texttt{v})来取代\texttt{call/cc}表达式进行构建,
然后用\texttt{v}对结果进行抽象:\texttt{k\^{} = (lambda\^{} (v) (+ 3 v))}。

\subsection{一个\texttt{call/cc}的简单例子}
\begin{minted}[mathescape=true,
               linenos,
               numbersep=5pt,
               autogobble,
               frame=lines,
               fontsize=\small,
               bgcolor=bg,
               framesep=2mm]{Racket}
    (+ (call/cc
          (lambda (k^)
            (/ (k^ 5) 4)))
       8)
\end{minted}
$\Longrightarrow$ 首先构建 \texttt{k\^{} = (lambda\^{} (v) (+ v 8))}\\
$\Longrightarrow$ 用\texttt{k\^{}}对\texttt{(/ (k\^{} 5) 4)}进行求值\\
$\Longrightarrow$ 对\texttt{(k\^{} 5)}进行求值\\
$\Longrightarrow$ 由于\texttt{k\^{}}是逃逸过程,除法过程被丢掉。最终得到结果13。

\subsection{延续描述的是余下的计算部分}
\noindent{}让我们通过一个例子来说明这件事:
\begin{minted}[mathescape=true,
               linenos,
               numbersep=5pt,
               autogobble,
               frame=lines,
               fontsize=\small,
               bgcolor=bg,
               framesep=2mm]{Racket}
    (* (+ (call/cc
            (lambda (k^)
              (/ (k^ 5) 4)))
          8)
       3)
\end{minted}
$\Longrightarrow$ 首先构建 \texttt{k\^{} = (lambda\^{} (v) (* (+ v 8) 3))}\\
$\Longrightarrow$ 对 \texttt{(/ (k\^{} 5) 4)}求值
$\Rightarrow$ 对 \texttt{(k\^{} 5)}求值\\
$\Longrightarrow$ 对 \texttt{((lambda\^{} (v) (* (+ v 8) 3)) 5)}求值\\
$\Longrightarrow$ 得到结果39。

\vspace{0.5em}
\noindent{}延续描述的是接下来还要做的事情,在上面这个例子里,对应于很直接的算数运算。实际上,接下来要做的事情,
或者说余下的计算,可以是任意定义的。我们再展示一个稍微复杂一点的例子:
\begin{minted}[mathescape=true,
               linenos,
               numbersep=5pt,
               autogobble,
               frame=lines,
               fontsize=\small,
               bgcolor=bg,
               framesep=2mm]{Racket}
    (* (+ (let ([u (+ 3 2)])
            (call/cc
              (lambda (j^)
                (/ (j^ u) 4)))) 8) 3)
\end{minted}
\noindent{}在这个例子里,\texttt{let}部分的``\texttt{u} 为 5''发生于确定\texttt{j\^{}}之前。该例中的延续\texttt{j\^{}}
与上面例子里的\texttt{k\^{}}是相同的。因为\texttt{(+ 3 2)}是 5,所以这两个例子给出了相同的结果。

\subsection{延续是第一类对象(first-class objects)}
\indent{}利用\texttt{call/cc},我们可以选择将逃逸过程存起来,直到在后面的计算中才去调用它。这里我们再来看一个例子,
这在个例子中我们将逃逸过程像变量一样存起来,存起来后可以在后面使用。
\begin{minted}[mathescape=true,
               linenos,
               numbersep=5pt,
               autogobble,
               frame=lines,
               fontsize=\small,
               bgcolor=bg,
               framesep=2mm]{Racket}
    (+ (call/cc
         (lambda (k^)
           (begin
             (set! +8^ k^)
             (display "inside body")
             5)))
       8)
\end{minted}
$\Longrightarrow$ 首先构建 \texttt{k\^{} = (lambda\^{} (v) (+ v 8))}\\
$\Longrightarrow$ 对 \texttt{(begin ...)}部分进行求值\\
$\Longrightarrow$ 词法变量 \texttt{+8\^{}}被置为 \texttt{(lambda\^{} (v) (+ v 8))}\\
$\textrm{$\;\;$}\quad$ 字符串\texttt{"inside body"} 被打印\\
$\textrm{$\;\;$}\quad$ 值5被作为\texttt{(begin ...)}部分的结果返回\\
$\Longrightarrow$ 整个表达式的结果是13。

\vspace{0.5em}
\noindent{}之后我们还可以使用\texttt{+8\^{}},比如\texttt{(* (/ (+8\^{} 35) 0) 100)}会得到43。

\section{元编程(Meta-Programming)}
\subsection{\texttt{call/cc}的默认法则}
\indent{}在之前的例子中,之所以5会被送给等待中的\texttt{+},是因为我们说:
\begin{displaymath}
    \textrm{\texttt{(call/cc (lambda (k\^{}) e))}} = \textrm{\texttt{(call/cc (lambda (k\^{}) (k\^{} e)))}}
\end{displaymath}
也就是说,等待结果的延续与\texttt{call/cc}所创建的逃逸过程是相同的。从这个等式中,我们可以知
道\texttt{e}默认为\texttt{(k\^{} e)}。因而,下面这个例子的结果是\texttt{(k\^{} 5)},仍为13。
\begin{minted}[mathescape=true,
               linenos,
               numbersep=5pt,
               autogobble,
               frame=lines,
               fontsize=\small,
               bgcolor=bg,
               framesep=2mm]{Racket}
    (+ (call/cc
         (lambda (k^)
           (k^ (begin
                 (set! +8^ k^)
                 (display "inside body")
                 5))))
       8)
\end{minted}

\subsection{\texttt{call/cc}的默认法则对\texttt{lambda\^{}}不成立}
\indent{}通过将\texttt{call/cc}应用到一个逃逸过程(即一个\texttt{(lambda\^{} ...)}过程)上的这种逃逸,会强加给用户
更多的控制责任,用户需要对``从被应用的逃逸过程的程序体中退出''做出决策。我们来看一看下面的两个例子,以对
这件事情有更多地理解:
\begin{align*}
&\textrm{\texttt{(+ 3 (call/cc (lambda\^{} (k\^{}) (k\^{} 8))))}} \quad\rightarrow \quad\textrm{Here we would get: 11}\\
&\textrm{\texttt{(+ 3 (call/cc (lambda\^{} (k\^{}) 8)))}} \quad\rightarrow \quad \textrm{But here we would just get: 8}
\end{align*}

\indent{}如果使用我们上面提到的\texttt{call/cc}的默认法则,似乎两个例子应该是等价的。不过,需要注意上面的法则中用到的是
\texttt{(lambda ...)} 而非 \texttt{(lambda\^{} ...)}。需要记住的是,\texttt{call/cc}可能会作用于常规过程或逃逸过程,
而延续总是一个类似于\texttt{k\^{}}的逃逸过程。

\subsection{一个简单的类-LISP的\texttt{BREAK}}
\indent{}通过将延续存下来,我们可以写出一个\texttt{BREAK} 过程,并可让用户自行决定计算的\texttt{RESUME}。当调用\texttt{BREAK}
时,一个消息会被返回至顶层。之后,系统的监听循环将处于受控状态,直到用户执行\texttt{(RESUME ...)},在那时送往\texttt{RESUME}
的参数是原本\texttt{(BREAK ...)} 调用的值。
\begin{minted}[mathescape=true,
               linenos,
               numbersep=5pt,
               autogobble,
               frame=lines,
               fontsize=\small,
               bgcolor=bg,
               framesep=2mm]{Racket}
    (define BREAK
      (lambda (message)
        (call/cc
          (lambda (k^)
            (set! RESUME k^)
            ((lambda^ (x) x) message)))))
\end{minted}
\noindent{}\texttt{BREAK}是一个通过使用一级延续得到优雅解决方案的十分有趣的例子。Brian Smith 选择了一个更为复杂
的\texttt{BREAK}版本作为反射(reflection)的例子%
\footnote{Smith, Brian Cantwell. Reflection and semantics in Lisp. ACM POPL, 1984, 23-35.}。

\subsection{如何构建\texttt{lambda\^{}}}
\indent{}让我们来考虑如何构建\texttt{(lambda\^{} (i ...) e ...)},并注意为何我们不能将\texttt{(lambda\^{} ...)}看作Scheme的
一部分,而必须总是在运行时构建它。我们定义一个过程\texttt{INVOKE/NO-CONT},它取一个无参过程\texttt{f}作为参数,并像调用
一个逃逸过程一样调用\texttt{f}过程。那么可以将\texttt{lambda\^{}}表示为:
\begin{align*}
&\textrm{\texttt{(lambda\^{} (id ...) e ...)}}\\ &\;\;\equiv \\
&\textrm{\texttt{(lambda (id ...) (INVOKE/NO-CONT (lambda () e ...)))}}
\end{align*}
\noindent{}下面是\texttt{make-INVOKE/NO-CONT}过程:
\begin{minted}[mathescape=true,
               linenos,
               numbersep=5pt,
               autogobble,
               frame=lines,
               fontsize=\small,
               bgcolor=bg,
               framesep=2mm]{Racket}
    (define make-INVOKE/NO-CONT
      (lambda ()
        ((call/cc
           (lambda (k^)
             (set! INVOKE/NO-CONT (lambda (th) (k^ th)))
             (lambda () 'INVOKE/NO-CONT))))))
\end{minted}
\indent{}必须在顶层调用\texttt{(make-INVOKE/NO-CONT)}。可以看到,在\texttt{INVOKE/NO-CONT}的构建中,\texttt{k\^{}}实际上
是不带延续的\texttt{(lambda\^{} (v) (v))}。因此,我们是在“空的”或“顶层”延续中调用一个无参过程(注意到包围\texttt{call/cc}
的双层括号)。通过将\texttt{INVOKE/NO-CONT}绑定到一个唯一目的是调用其参数的过程上,我们可以保证没有延续在等待它的结果。

\subsection{延续传递风格(continuation-passing-style)}
\indent{}下面的程序会给出一个二叉搜索树所有节点值之和。另外,如果给定的二叉搜索树中的某个节点
的值为0,则会给出0作为最终结果。这里我们使用\texttt{call/cc}来产生逃逸过程,并在发现某个节点的值为零时直接
返回最终结果,通过\texttt{exit\^{}}跳出来。
\begin{minted}[mathescape=true,
               linenos,
               numbersep=5pt,
               autogobble,
               frame=lines,
               fontsize=\small,
               bgcolor=bg,
               framesep=2mm]{Racket}
    (define sum-bst
      (lambda (t)
        (call/cc
          (lambda (exit^)
            (letrec
              ([sum-bst
                 (lambda (t)
                   (cond
                     [(null? t) 0]
                     [(zero? (info t)) (exit^ 0)]
                     [else (+ (info t)
                              (sum-bst (left t))
                              (sum-bst (right t)))]))])
              (sum-bst t))))))
\end{minted}
\indent{}上述过程可以被等价地用延续传递风格进行改写,如下程序片段所示。这样的程序经过练习就可以读懂和写出。
具体分析一下,考虑\texttt{else} 那一行,它的意思是:想象你有左子树的对应结果并记为\texttt{result1},然后想象你有右子树
对应的结果并记为\texttt{result2},之后取根节点的值与\texttt{result1}和\texttt{result2}之和,并将其传给
等待中的延续\texttt{k}。零检测那一行意味着抛弃等待着的延续,并将值0返回给\texttt{bst}原始调用者的延续。
这可能有点别扭,但真正困难之处在于这种风格具有“感染性”,也就是说,在使用这种风格的同时也会将其意志强加给每个
程序。举个例子,如果使用了一个映射过程,那么要被映射的过程也会需要是以这种风格写成的。
\begin{minted}[mathescape=true,
               linenos,
               numbersep=5pt,
               autogobble,
               frame=lines,
               fontsize=\small,
               bgcolor=bg,
               framesep=2mm]{Racket}
    (define sum-bst
      (lambda (t)
        (letrec
          ([sum-bst
             (lambda (t k)
               (cond
                 [(null? t) (k 0)]
                 [(zero? (info t)) 0]
                 [else (sum-bst (left t)
                         (lambda (result1)
                           (sum-bst (right t)
                             (lambda (result2)
                               (k (+ (info t) result1 result2))))))]))])
          (sum-bst t (lambda (x) x)))))
\end{minted}
\indent{}引用James S. Miller的话\footnote{Miller, James S. Multischeme: A Parallel Processing System
Based on MIT Scheme. MIT Ph.~D dissertation, 1987.}:“不幸的是,通过转换过程产生的程序往往是难以理解的。关于不需要在
Scheme中加入延续的论点实际上是正确的。它和这句话一样,都是有道理的:形式参数的名称可以是任选的。而且,这些论点
都有着相同的基本瑕疵:一条陈述被写下的形式对人们是否容易理解它是有重大影响的。尽管理解到语言不需要固有地对
延续提供任何支持,Scheme社区仍然选择在语言中加入一个操作来减轻繁琐性。”

\subsection{用\texttt{call/cc}做元编程:\texttt{CYCLE}}
\noindent{}\texttt{CYCLE}与\texttt{EXIT-CYCLE-WITH}一起用于无限循环。
\begin{minted}[mathescape=true,
               linenos,
               numbersep=5pt,
               autogobble,
               frame=lines,
               fontsize=\small,
               bgcolor=bg,
               framesep=2mm]{Racket}
    (define CYCLE
      (lambda (f)
        (call/cc
          (lambda (k^)
            (letrec ([loop (lambda ()
                             (f k^)
                             (loop))])
              (loop))))))
\end{minted}
\noindent{}\texttt{CYCLE}的协议满足:
\begin{displaymath}
\textrm{\texttt{(CYCLE (lambda (EXIT-CYCLE-WITH) e ...))}}
\end{displaymath}
\noindent{}期望是在\texttt{e ...}中的某处有至少一次\texttt{(EXIT-CYCLE-WITH ...)}调用。这样一来,无限循环
以遇到的第一个\texttt{(EXIT-CYCLE-WITH ...)}结束。

\subsection{带有\texttt{GO}与\texttt{RETURN}的LISP之\texttt{PROG}:“GO TO”编程再探}
\noindent{}让我们考虑一下LISP的\texttt{PROG}:
\begin{minted}[mathescape=true,
               %linenos,
               numbersep=5pt,
               autogobble,
               frame=lines,
               fontsize=\small,
               bgcolor=bg,
               escapeinside=||,
               framesep=2mm]{text}
    (PROG (id ...)
      label|$_1$| e|$_1$| ...
      ...
      label|$_{n-1}$| e|$_{n-1}$| ...
      label|$_{n}$| e|$_{n}$| ...)
    |$\equiv$|
    (let ([id '()] ...)
      ((call/cc
        (lambda (GO)
          (let ([RETURN (lambda (v) (GO (lambda () v)))])
            (letrec
              ([label|$_1$| (lambda () e|$_1$| ... (label|$_2$|))]
               ...
               [label|$_{n-1}$| (lambda () e|$_{n-1}$| ... (label|$_n$|))]
               [label|$_n$| (lambda () e|$_n$| ... (RETURN '()))])
              (label|$_1$|)))))))
\end{minted}
\noindent{}这样的定义让我们避免任何丢失尾递归的风险。即便是包含怪异代码的程序,比如
\begin{minted}[mathescape=true,
               linenos,
               numbersep=5pt,
               autogobble,
               frame=lines,
               fontsize=\small,
               bgcolor=bg,
               framesep=2mm]{Racket}
    (begin
      (GO x)
      (print 3))
\end{minted}
\noindent{}它仍会像是只有\texttt{(GO x)}那样执行。我们仅需要调用\texttt{(label$_i$)}而非\texttt{(GO label$_i$)}的
原因是这些特别的调用总是处于尾递归的位置。

\indent{}严格来说这些定义比必要的更具一般性。比如说,在\texttt{e ...}中可能有\texttt{(f label$_i$)},\texttt{(f RETURN)},
和\texttt{(f GO)}这样的实例。并且,\texttt{label$_i$},\texttt{RETURN}和\texttt{GO}可能会被存起来以备后用。这就是
一些用于非盲(non-blind)回溯的机制,例如\texttt{a-bien-tot}与\texttt{au-revoir}的实现方式。

\section{同步进程(Synchronous Processes)}

\subsection{延续作为同步进程}
\indent{}我们接下来从面向进程的角度来审视延续。一个延续表示了控制点,我们可以用它来实现进程。一个进程调度器
可以基于延续的就绪队列(ready queue)来设计。

\indent{}如果我们考虑一列表达式\texttt{(begin S$_1$ S$_2$ ...)},那么无论\texttt{S$_i$}何时绑定到一个延续,
它的延续看起来像是\texttt{(lambda\^{} (waste) S$_{i+1}$ S$_{i+2}$ ...)}。它们被称为
命令延续(command continuations),因为我们对延续调用的值本身并不感兴趣。我们选用名字“waste”来展示这一点。
举个例子,对于:
\begin{minted}[mathescape=true,
               linenos,
               numbersep=5pt,
               autogobble,
               frame=lines,
               fontsize=\small,
               bgcolor=bg,
               framesep=2mm]{Racket}
    (define foo
      (lambda ()
        (display 2)
        (call/cc (lambda (ak^) ...))
        (display 3)
        (call/cc (lambda (bk^) ...))
        (display 5)
        (call/cc (lambda (ck^) ...))
        (display 7)))
\end{minted}
\noindent{}如果我们在顶层调用\texttt{foo}(没有上下文),那么:
\begin{minted}[mathescape=true,
               linenos,
               numbersep=5pt,
               autogobble,
               frame=lines,
               fontsize=\small,
               bgcolor=bg,
               framesep=2mm]{text}
    ak^ = (lambda^ (waste)
            (display 3)
            (call/cc (lambda (bk^) ...))
            (display 5)
            (call/cc (lambda (ck^) ...))
            (display 7))

    bk^ = (lambda^ (waste)
            (display 5)
            (call/cc (lambda (ck^) ...))
            (display 7))

    ck^ = (lambda^ (waste)
            (display 7))
\end{minted}

\subsection{语言的扩展}
\indent{}我们扩展语言,并引入\texttt{(PAUSE-HANDLER)}。如果在我们的例子中
将\texttt{(call/cc (lambda (ik\^{}) ...))}用一个过程调用\texttt{PAUSE-HANDLER}取代,我们就得到像这样的片段:
\begin{minted}[mathescape=true,
               linenos,
               numbersep=5pt,
               autogobble,
               frame=lines,
               fontsize=\small,
               bgcolor=bg,
               framesep=2mm]{Racket}
    (define foo
      (lambda ()
        (display 2)
        (PAUSE-HANDLER)
        (display 3)
        (PAUSE-HANDLER)
        (display 5)
        (PAUSE-HANDLER)
        (display 7)))
\end{minted}
\noindent{}那么,\texttt{PAUSE-HANDLER}的过程就会负责\texttt{call/cc}。
\begin{minted}[mathescape=true,
               linenos,
               numbersep=5pt,
               autogobble,
               frame=lines,
               fontsize=\small,
               bgcolor=bg,
               framesep=2mm]{Racket}
    (define PAUSE-HANDLER
      (lambda ()
        (call/cc
          (lambda (k^)
            ... ...))))
\end{minted}
\noindent{}剩下就是决定要用已经绑到\texttt{call/cc}上的延续去做的事情。在这个简单的例子里,我们会:
\begin{enumerate}
\item 将延续插入进程队列
\item 从队列前端删掉\texttt{k\^{}}(即进程对象)
\item 调用延续\texttt{k\^{}}
\end{enumerate}
因而我们将\texttt{PAUSE-HANDLER}定义为:
\begin{minted}[mathescape=true,
               linenos,
               numbersep=5pt,
               autogobble,
               frame=lines,
               fontsize=\small,
               bgcolor=bg,
               framesep=2mm]{Racket}
    (define PAUSE-HANDLER
      (lambda ()
        (call/cc
          (lambda (k^)
            (swap-run the-process-q k^)))))
\end{minted}

\subsection{交换和执行当前进程}
\noindent{}\texttt{swap-run}会做上面提到的三件事。
\begin{minted}[mathescape=true,
               linenos,
               numbersep=5pt,
               autogobble,
               frame=lines,
               fontsize=\small,
               bgcolor=bg,
               framesep=2mm]{Racket}
    (define swap-run
      (lambda (q k^)
        (q 'en-q! k^)
          (RUN (q 'de-q!))))

    (define RUN (lambda (k^) (k^ 'waste)))
\end{minted}
\noindent{}\texttt{RUN}调用从队列前端被取出的延续。

\subsection{创建进程}
现在留下的仅有的问题是建立用于进程的协议:“使用\texttt{PAUSE-HANDLER}的代码”。
当没有东西要处理的时候,我们应该怎么办呢?一个简单的解决方法是要求插入一个\texttt{HALT}命令。
我们通过引入\texttt{CREATE-PROCESS}阐明这个协议。进程,就像从其中刻画出来的延续一样,必须是带有
一个参数的过程。
\begin{minted}[mathescape=true,
               linenos,
               numbersep=5pt,
               autogobble,
               frame=lines,
               fontsize=\small,
               bgcolor=bg,
               framesep=2mm]{Racket}
    (define CREATE-PROCESS
      (lambda (th)
        (lambda (v)
          (th)
          (HALT))))
\end{minted}

\subsection{\texttt{HALT}与\texttt{DISPATCHER}}
\indent{}我们接下来定义\texttt{DISPATCHER}过程,将其作为一个简单的同步进程调度器。每当\texttt{DISPATCHER}被
调用,一个新的队列就被创建出来。通过\texttt{(create-q exit\^{})}创建的队列用\texttt{exit\^{}}延续构建。
无论何时试图从空队列删除一个进程时,队列会调用\texttt{exit\^{}}延续,这将退出\texttt{DISPATCHER}。如果
队列并不为空,我们就成功地从队列中删除一个元素并\texttt{RUN}它。我们将\texttt{PAUSE-HANDLER}包含进\texttt{DISPATCHER}
的定义中,这样它就会覆盖自由变量\texttt{the-process-q}。
\begin{minted}[mathescape=true,
               linenos,
               numbersep=5pt,
               autogobble,
               frame=lines,
               fontsize=\small,
               bgcolor=bg,
               framesep=2mm]{Racket}
    (define DISPATCHER
      (lambda (initialize-q)
        (call/cc
          (lambda (exit^)
            (let ([the-process-q (create-q exit^)])
              (set! HALT
                (lambda ()
                  (RUN (the-process-q 'de-q!))))
              (set! PAUSE-HANDLER
                (lambda ()
                  (call/cc
                    (lambda (k^)
                      (swap-run the-process-q k^)))))
              (initialize-q the-process-q)
              (HALT))))))
\end{minted}
\noindent\texttt{DISPATCHER}的调用者将一个取一个队列为参数的过程传入。这个过程在第一个进程\texttt{RUN}ning
之前被调用。我们使用它来初始化队列。

\section{附录B:\texttt{DISPATCHER}的支持代码}
\indent{}在附录B中,我们会展示如何使用\texttt{initialize-q}过程,并提供执行\texttt{DISPATCHER}所必须的全部代码。
由\texttt{create-q}创建的队列会对消息\texttt{en-q!}与\texttt{de-q!}进行响应。如果尝试从一个空的队列删除
东西,那么作为\texttt{create-q}的参数被传入的延续会被调用。如果有入队操作在先,那么测试空队列就是不需要的。
一个好的实践是添加将\texttt{en-q!}和\texttt{de-q!}动作结合而成的新的消息。那么\texttt{swap-run}就成为了
\texttt{(RUN (q 'en-q-de-q! k\^{}))}。
\begin{minted}[mathescape=true,
               linenos,
               numbersep=5pt,
               autogobble,
               frame=lines,
               fontsize=\footnotesize,
               bgcolor=bg,
               framesep=2mm]{Racket}
(define create-q
  (lambda (where-to-exit-when-empty^)
    (let ([head '()] [tail t])
      (lambda msg
        (case (car msg)
          [en-q! (if (null? head)
                     (begin (set! head (cons (cadr msg) head))
                            (set! tail head))
                     (begin (set-cdr! tail (cons (cadr msg) '()))
                            (set! tail (cdr tail))))]
          [de-q! (if (null? head)
                     (where-to-exit-when-empty^ 'done)
                     (begin0 (car head)
                             (if (eq? head tail)
                                 (set! head '())
                                 (set! head (cdr head)))))])))))
\end{minted}
\noindent{}过程\texttt{process-maker},\texttt{build-q}和\texttt{test}创建了一个测试程序。
\begin{minted}[mathescape=true,
               linenos,
               numbersep=5pt,
               autogobble,
               frame=lines,
               fontsize=\footnotesize,
               bgcolor=bg,
               framesep=2mm]{Racket}
(define process-maker
  (lambda (n)
    (CREATE-PROCESS
      (lambda ()
        (display n)
        (PAUSE-HANDLER)
        (newline)
        (display "about to halt")
        (newline)
        (PAUSE-HANDLER)
        (display n)))))

(define build-q
  (lambda (maker n)
    (lambda (q)
      (letrec
        ([loop (lambda (n)
                 (cond
                  [(zero? n) 'queue-built]
                  [else (q 'en-q! (maker n))
                        (loop (sub1 n))]))])
        (loop n)))))

(define test
  (lambda ()
    (DISPATCHER (build-q process-maker 8))))
\end{minted}

\end{document}