\documentclass[12pt]{article}
\usepackage{array}
\usepackage{amsmath}
\usepackage{amssymb}
\usepackage{xfrac}
\usepackage{ntheorem}
\usepackage{algorithm}
\usepackage{algorithmic}
\usepackage{caption}
\usepackage{fontspec}
\usepackage{graphicx}
\usepackage{indentfirst}
\usepackage{enumitem}
\usepackage{minted}
\usepackage{mathtools}
\usepackage{pifont}
\usepackage{setspace}
\usepackage{subfigure}
\usepackage{tikz}
\usepackage{url}
\usepackage{tcolorbox}
\usepackage{xcolor}
\usepackage{xeCJK}

\usepackage[colorlinks=true]{hyperref}
\usepackage[margin=0.55in]{geometry}

% background color for minted
\definecolor{bg}{rgb}{0.95,0.95,0.95}

% CJK font
\setCJKmainfont{Source Han Serif CN}

% indent value
\setlength{\parindent}{2em}

% line spacing
\linespread{1.2}
\renewcommand{\thesection}{\Roman{section}}
\renewcommand{\thesubsection}{\thesection-\arabic{subsection}}

\title{All About Monads}
\author{Yiteng Zhang}

\begin{document}
\maketitle

\noindent{}\textbf{注:}时至今日,网络上有关monad的介绍数不胜数,大都冠以“理解monad”此类标题。多读文章
定是有所助益的,但若能在那之前将基本概念了解透彻,必能事半功倍。Haskell Wiki提供的tutorial文章
\textit{All About Monads}(关于单子的一切)就是这样一篇能够帮助我们稳固基础的上好材料。这篇文章是对
它的简译,以便日后的回顾与研习。原文较长,这里不求详尽,仅涵盖“monad变换器剖析”一节之前的内容。其后
的大量例子对monad变换器有更深入的讨论,直接阅读原文较好。

\vspace{0.5em}
\noindent{}这篇tutorial介绍了monad和monad变换器(monad transformer)的相关概念,并带领我们观览了
很多常用的monad。文章主要分为三个部分:第一部分是对monad基本概念的介绍,并说明了Haskell对monad的支持;
第二部分重点介绍了九种常用的monad,并分别给出了相关例子;第三部分介绍了monad在现实中的应用,并针对monad变换器进行了阐述。

\vspace{-0.5em}
\section{理解Monads}
\noindent{\textbf{什么是monad?}}Monad是一种根据值和使用这些值的计算序列来构造计算的一种方式。Monads能够让
我们使用顺序化组件来建立计算,而组件本身也可以是计算序列。Monad决定了如何将几个计算结合起来以形成一个新的
计算,这使得我们不必在每次需要它时都手写组合代码。

\indent{}将一个monad想作一种用于把一些计算组合成一个复杂计算的策略是很有帮助的。例如,对于我们所熟知的Haskell中
的\texttt{Maybe}类型:
\begin{minted}[mathescape=true,
               %linenos,
               numbersep=5pt,
               autogobble,
               %frame=lines,
               fontsize=\footnotesize,
               bgcolor=bg,
               framesep=2mm]{Haskell}
data  Maybe a  =  Nothing | Just a
\end{minted}
它表示的是可能会失败从而无法返回结果的计算类型。\texttt{Maybe}类型给出了一种用于组合返回\texttt{Maybe}值的计算的策略:
如果组合后的计算包含一个计算\texttt{B},这个计算\texttt{B}又依赖于另一个计算\texttt{A}的结果,那么无论是\texttt{A}还
是\texttt{B}给出了\texttt{Nothing},组合后的计算都会给出\texttt{Nothing}。如果两个计算\texttt{A}和\texttt{B}都是成功的,
那么组合后的计算给出的结果会是\texttt{B}应用在\texttt{A}的结果上产生的结果。

\indent{}还有很多不同种类的monad,它们用于不同的计算组合策略。其中,一些常见而又特别有用的monads被包含进了Haskell 2010标准
库中。我们会在第二节详细介绍他们。
~\\

\noindent{\textbf{为什么应该尽量去理解monads?}}现在网络上有大量不同的monad教程,这充分说明了很多人对于理解这个概念是
有困难的。之所以会有困难,是因为monad本身的抽象性,以及它们被用作多种不同的用途,这可能会让人对monad的具体概念感到迷惑。

\indent{}在Haskell中,monad是I/O系统的关键。要在Haskell做I/O,并不一定非要去理解monad才行,但是对I/O monad的理解会改善你
的代码,并扩充你的能力。对于编程者来说,monads是构造函数式程序的有用工具,它们的有用性体现在三个性质上:
\begin{itemize}
\item 模块性(Modularity):它们允许计算由一些更加简单的计算合并而成,并将合并策略与实际正在进行的计算分离开。
\item 灵活性(Flexibility):比起不使用monads,它们能让函数式程序的适用性更强。这是因为monad将计算策略抽取并独立放置,而非
让它分布在整个程序中。
\item 隔离性(Isolation):他们可以用来创建命令式( imperative-style)的计算结构,这些结构可以安全地与函数式程序的主体隔离。
这有助于将副作用(比如I/O)与状态(会打破引用透明性)并入像Haskell这样的纯函数式语言中。
\end{itemize}
\noindent{}在后续部分中,我们会结合具体的monad对以上每一个特性进行讨论。

\subsection{认识Monads}
\subsubsection{类型构造器(Type constructors)}
为了理解Haskell中的monads,需要先熟悉类型构造器。类型构造器是一个参数化的类型定义,用于多态类型。通过向一个
类型构造器提供一个或多个具体的类型(作为其参数),我们可以创建一个新的具体的类型。例如,在\texttt{Maybe}的定义中,
\texttt{Maybe}即是一个类型构造器,而\texttt{Nothing}与\texttt{Just}是数据构造器(data constructors)。我们可以
通过将数据构造器应用到值上来创建一个数据值(data value)。
\begin{minted}[mathescape=true,
               %linenos,
               numbersep=5pt,
               autogobble,
               %frame=lines,
               fontsize=\footnotesize,
               bgcolor=bg,
               framesep=2mm]{Haskell}
country  =  Just "China"
\end{minted}
\noindent{}作为一个例子,上面这行代码中,我们将数据构造器\texttt{Just}应用到字符串值\texttt{"China"}上,构建了
一个data value \texttt{country}。类似地,我们可以将类型构造器应用到类型上来构建新的类型:
\begin{minted}[mathescape=true,
               %linenos,
               numbersep=5pt,
               autogobble,
               %frame=lines,
               fontsize=\footnotesize,
               bgcolor=bg,
               framesep=2mm]{Haskell}
lookupAge :: DB -> String -> Maybe Int
\end{minted}
\indent{}多态类型就像一个容器,可以容纳许多不同类型的值。所以\texttt{Maybe Int}可以被看作是容纳\texttt{Int}类型的值
的\texttt{Maybe}容器,而\texttt{Maybe String}可被看作是容纳\texttt{String}类型值的\texttt{Maybe}容器。在Haskell中,我们
还能让容器的类型也是多态的,所以我们可以写出\texttt{m a}来表示一个具有某种类型\texttt{m}的,
且容纳某种类型\texttt{a}的值的容器。

\indent{}我们经常协同使用类型构造器与类型变量(type variable)来描述一个计算的抽象特征。比如,多态类型\texttt{Maybe a}是
所有可能返回一个值或\texttt{Nothing}的计算的类型。利用这种方式,我们就能够脱离容器所容物的细节来讨论容器的属性。

\subsubsection{Maybe a monad}
在Haskell中,一个monad被表示为一个类型构造器(我们称其为\texttt{m}),一个用于构建具有那种类型的值
的函数(称为\texttt{a -> m a}),以及一个将该类型的值与产生该类型值的计算相结合以产生一个
新的计算(这个计算也产生该类型的值)的函数(称为\texttt{m a -> (a -> m b) -> m b})。要注意的是,容器是相同的,
但容器所容纳内容的类型是可以变的。习惯上,在讨论monad时,我们称monad类型构造器为\texttt{m}。用于构建具有那种类型的值的
函数被称为\texttt{return}。剩下的那个函数被称为bind,但通常写作\texttt{>>=}(它用作中缀函数)。这些函数的签名如下所示:
\begin{minted}[mathescape=true,
               %linenos,
               numbersep=5pt,
               autogobble,
               frame=lines,
               fontsize=\footnotesize,
               bgcolor=bg,
               framesep=2mm]{Haskell}
-- the type of monad m
data  m a  =  ...

-- return takes a value and embeds it in the monad
return :: a -> m a

-- bind is a function that combines a monad instance m a with a computation
-- that produces another monad instance m b from a's to produce a new monad instance m b
(>>=) :: m a -> (a -> m b) -> m b
\end{minted}
\indent{}粗略地讲,monad类型构造器定义了一个计算的类型,函数\texttt{return}创建那个计算类型的基元值(primitive values),
而\texttt{>>=}将那种类型的计算合并在一起来创建一个相同类型的更为复杂的计算。用容器类比的话,类型构造器\texttt{m}是
一个容纳不同值的容器。\texttt{m a}是一个容纳类型为\texttt{a}的值的容器。\texttt{return}将一个值推入一个monad容器。
\texttt{>>=}从一个monad容器中取出值,并将其送入一个用于产生一个包含新值的monad容器中(新值的类型与之前取出的值的类型可以不一样)。
\texttt{>>=}被称为bind,因为它将一个monad中的值绑定到一个函数的第一个参数上。通过赋予bind函数具体的逻辑,一个monad可以为
计算的组合实现特定的策略。

\subsubsection{一个例子}
\indent{}我们用一个例子来展示一下上一小节所阐述的概念。假设我们正在编写一个程序来跟踪克隆羊实验。
我们当然想知道所有羊的遗传史,所以我们需要母羊和父羊函数。
但由于这些羊是克隆羊,它们不一定总有母羊和父羊!我们可以用\texttt{Maybe}来表示可能没有父/母羊的可能性:
\begin{minted}[mathescape=true,
               %linenos,
               numbersep=5pt,
               autogobble,
               frame=lines,
               fontsize=\footnotesize,
               bgcolor=bg,
               framesep=2mm]{Haskell}
type  Sheep  =  ...

father :: Sheep -> Maybe Sheep
father = ...

mother :: Sheep -> Maybe Sheep
mother = ...
\end{minted}
\noindent{}之后,定义函数来找到祖辈的羊。这可能会有点复杂,因为可以没有父羊或母羊:
\begin{minted}[mathescape=true,
               %linenos,
               numbersep=5pt,
               autogobble,
               %frame=lines,
               fontsize=\footnotesize,
               bgcolor=bg,
               framesep=2mm]{Haskell}
maternalGrandfather    :: Sheep -> Maybe Sheep
maternalGrandfather s  =  case (mother s) of
                            Nothing -> Nothing
                            Just m  -> father m
\end{minted}
\noindent{}其它的寻找祖辈的函数可以用类似的方式定义。我们还能进一步写出寻找曾祖辈羊的函数:
\begin{minted}[mathescape=true,
               %linenos,
               numbersep=5pt,
               autogobble,
               %frame=lines,
               fontsize=\footnotesize,
               bgcolor=bg,
               framesep=2mm]{Haskell}
mothersPaternalGrandfather    :: Sheep -> Maybe Sheep
mothersPaternalGrandfather s  =  case (mother s) of
                                   Nothing -> Nothing
                                   Just m  -> case (father m) of
                                                Nothing -> Nothing
                                                Just gf -> father gf
\end{minted}

\indent{}除了丑陋、不清晰、难以维护外,这些实现还非常地费功夫。很明显,在计算的任何
一点上的\texttt{Nothing}值都会导致最后返回\texttt{Nothing}。如果能把这一概念单另实现,
并去掉散落在代码中的所有显式的\texttt{case}测试,就会好很多。这将使代码更容易编写,更容易阅读,也更易于修改。
所以好的编程风格会让我们创建一个组合器来捕获我们想要的行为:
\begin{minted}[mathescape=true,
               %linenos,
               numbersep=5pt,
               autogobble,
               frame=lines,
               fontsize=\footnotesize,
               bgcolor=bg,
               framesep=2mm]{Haskell}
-- comb is a combinator for sequencing operations that return Maybe
comb             :: Maybe a -> (a -> Maybe b) -> Maybe b
comb Nothing  _  =  Nothing
comb (Just x) f  =  f x

-- now we can use `comb` to build complicated sequences
mothersPaternalGrandfather    :: Sheep -> Maybe Sheep
mothersPaternalGrandfather s  =  (Just s) `comb` mother `comb` father `comb` father
\end{minted}
\noindent{}组合器功效显著!上面的代码清晰简洁,易于理解与修改。注意\texttt{comb}函数是完全多态的,它并不
是专为\texttt{Sheep}所用的。实际上,组合器捕获到了“用于合并可能无法返回值的计算的一般策略”。因此,我们可以
将这个组合器应用到其它可能无法返回值的计算中,比如数据库查询或字典查找。

\indent{}不知不觉中,我们已经得到了一个monad。\texttt{Maybe}类型构造器与\texttt{Just}函数(起\texttt{return}的作用)和
我们的组合器(起\texttt{>>=}的作用)一起构成了一个简单的monad,这个monad用于构建可能不返回值的计算。在下一章中,我们
使其符合Haskell的monad框架以让它变得真正有用。

\subsubsection{列表(list)也是monad}
\indent{}我们已经看到,\texttt{Maybe}类型构造器是用于构建可能不返回值的计算的monad。你可能会惊讶于另一个
简单的Haskell类型构造器\texttt{[]}(用于构建列表),也是一个monad。所谓的list monad允许我们构建可以返回
零个,一个或多个值的计算。

\indent{}用于列表的\texttt{return}函数简单地创建一个单例(singleton)列表:\texttt{return x = [x]}。用于
列表的binding操作会创建一个新的列表,这个列表包含的是将函数应用到原列表的元素上所产生的结果。其定义为:
\texttt{l >>= f = concatMap f l}。

\indent{}返回列表的函数的一个用途是用来表示模糊(ambiguous)的计算,也就是可能会产生不确定数量返回值的计算。
在一个由模糊的子计算结合而成的计算中,其模糊性可能是复合的(compound),或者最终化为唯一允许的结果,或根本没有
可被接受的结果。在这一过程中,可能的计算状态的集合被表示为一个列表。因此,list monad体现了一种策略,这种策略
是用于这样一种计算:对于一个模糊的计算,沿着所有允许的路径同时进行计算。

\subsubsection{总结}
\indent{}如上所述,一个monad是:一个类型构造器,一个\texttt{return}函数,一个被称之为bind或\texttt{>>=}的组合器函数。
这三者一起作用,封装了一个用于将几个计算合并从而产生一个更为复杂的计算的策略。

\indent{}通过使用\texttt{Maybe}类型构造器,我们看到了良好的编程实践是如何引导我们去定义一个简单的monad的:该monad可
被用来从一个计算的序列中构建一个复杂的计算,这个计算序列中的每一个计算都是可能无法返回值的计算。由此产生的\texttt{Maybe}
monad封装了用于合并可能不返会值的计算的策略。通过将这种策略编入一个monad,我们获得了一定程度的模块性和灵活性。
\clearpage

\subsection{Doing it with class}
这一节的讨论会涉及到Haskell的类型类(type class)系统,因而在继续阅读之前可以先对
它进行回顾(注意区分type、class与typeclass的概念,typeclass亦是一种类,用\texttt{class}关键字来定义)。

\subsubsection{The Monad class}\label{subsec:minimal_definition}
\indent{}在Haskell中,有一个标准的\texttt{Monad}类,它定义了\texttt{return}和\texttt{>>=}这两个函数
的名称与签名。尽管不是非要让你的monads成为这个标准\texttt{Monad}类的实例不可,但这样做确实是个好主意。Haskell为
\texttt{Monad}实例提供了内建的特殊支持。让你的monads成为\texttt{Monad}的实例可以使你用上这些特性,从而
写出更为简洁优雅的代码。同时,让你的monads成为\texttt{Monad}类的实例实际上向代码阅读者传递了重要的信息,不这么
做的话可能会导致你使用带有歧义性或不标准的函数名。

\indent{}这个标准的\texttt{Monad}类看起来像这样:
\begin{minted}[mathescape=true,
               %linenos,
               numbersep=5pt,
               autogobble,
               %frame=lines,
               fontsize=\footnotesize,
               bgcolor=bg,
               framesep=2mm]{Haskell}
class Monad m where
    (>>=)  :: m a -> (a -> m b) -> m b
    return :: a -> m a
\end{minted}

\vspace{-0.7em}
\subsubsection{Example continued}
\indent{}让我们继续上一节提到的克隆羊的例子,我们将会看到\texttt{Maybe}类型构造器是如何作为\texttt{Monad}类
的一个实例融入Haskell的monad框架的。

\indent{}回想一下,我们的\texttt{Maybe} monad使用\texttt{Just}数据构造器来作monad的\texttt{return}函数之用,我们还
构建了一个简单的组合器来作monad的\texttt{>>=}函数之用。那么我们可以显式地将\texttt{Maybe}声明为\texttt{Monad}类的实例
使其作monad之用:
\begin{minted}[mathescape=true,
               %linenos,
               numbersep=5pt,
               autogobble,
               %frame=lines,
               fontsize=\footnotesize,
               bgcolor=bg,
               framesep=2mm]{Haskell}
instance Monad Maybe where
    Nothing  >>= f  =  Nothing
    (Just x) >>= f  =  f x
    return          =  Just
\end{minted}
\noindent{}一旦我们将\texttt{Maybe}定义为\texttt{Monad}类的一个实例,我们就能用标准的monad操作来构建复杂
的计算:
\begin{minted}[mathescape=true,
               %linenos,
               numbersep=5pt,
               autogobble,
               frame=lines,
               fontsize=\footnotesize,
               bgcolor=bg,
               framesep=2mm]{Haskell}
-- we can use monadic operations to build complicated sequences
maternalGrandfather    :: Sheep -> Maybe Sheep
maternalGrandfather s  =  (return s) >>= mother >>= father

fathersMaternalGrandmother    :: Sheep -> Maybe Sheep
fathersMaternalGrandmother s  =  (return s) >>= father >>= mother >>= mother
\end{minted}
\indent{}在Haskell标准的\texttt{prelude}中,\texttt{Maybe}是被定义为\texttt{Monad}类的一个实例的,所以你不用自己
来做这件事。至此我们看到的另一个monad,即列表构造器\texttt{[]},亦在\texttt{prelude}中被定义为\texttt{Monad}类的
一个实例。在使用monad写函数时,注意要使用\texttt{Monad}类,而不是使用特定的实例。
\begin{minted}[mathescape=true,
               %linenos,
               numbersep=5pt,
               autogobble,
               %frame=lines,
               fontsize=\footnotesize,
               bgcolor=bg,
               framesep=2mm]{Haskell}
doSomething :: (Monad m) => a -> m b
\end{minted}
\noindent{}这样的写法比写成 \mintinline{Haskell}|doSomething :: a -> Maybe b| 更灵活。

\indent{}前一种写法能够用于很多不同的monad类型,以根据对应monad所体现的策略来得到不同的行为。然而,后一种
写法将策略限制到了\texttt{Maybe} monad的策略,因而缺少了这层灵活性。

\subsubsection{Do notation}
\indent{}使用标准的monadic函数是很不错的一点,而加入\texttt{Monad}类的另一好处是Haskell对\texttt{do}记法的支持。
Do记法是一种表达上的简写,这种简写用于方便地构建monadic的计算,就像列表推导式是用于构建列表上的计算的一种简写。
\texttt{Monad}类的任意一个实例都可以用在Haskell的do block中。

\indent{}简单地讲,do记法使你能够以一种“伪”命令式的风格用命名变量来写monadic的计算。通过使用左箭头\texttt{<-}运算符,
一个monadic的计算可以被“赋值”给一个变量。之后,将这个变量用于后续的monadic计算中将会自动地进行binding操作。
在使用do记法时,箭头右侧的表达式的类型是一个monadic类型\texttt{m a},而箭头左侧的表达式是一个要与monad内部的
值进行匹配的模式(pattern)。

\indent{}这里有一个将\texttt{Maybe} monad用在do记法中的例子:
\begin{minted}[mathescape=true,
               %linenos,
               numbersep=5pt,
               autogobble,
               frame=lines,
               fontsize=\footnotesize,
               bgcolor=bg,
               framesep=2mm]{Haskell}
-- we can also use do-notation to build complicated sequences
mothersPaternalGrandfather    :: Sheep -> Maybe Sheep
mothersPaternalGrandfather s  =  do m  <- mother s
                                    gf <- father m
                                    father gf
\end{minted}
\noindent{}这个例子可以与上面的\texttt{fathersMaternalGrandmother}的写法对比着看。

\indent{}注意到do记法有点像一种命令式编程语言。就此而言,monads提供了在一个更大的函数式程序中创建命令式风格的
计算的可能性。稍后我们会在讨论副作用与I/O monad时扩展这个主题。

\indent{}Do记法只是语法糖罢了,不会有什么是能用do记法做而不能用标准monadic的操作做到的。不过,在一些情形下do记法
确实更加清晰和方便,特别是当monadic计算的序列非常长的时候。所以我们应该同时掌握标准的monadic的binding记法和
do记法,并在合适的时候选用它们。

\indent{}从do记法到标准的monadic操作的实际转换过程大致是,对于每个与模式\texttt{x <- expr1}相匹配的表达式,
都将其变成:
\begin{minted}[mathescape=true,
               %linenos,
               numbersep=5pt,
               autogobble,
               %frame=lines,
               fontsize=\footnotesize,
               bgcolor=bg,
               framesep=2mm]{Haskell}
expr1 >>= \x ->
\end{minted}
\noindent{}并且每个没有变量赋值的表达式\texttt{expr2},变成:
\begin{minted}[mathescape=true,
               %linenos,
               numbersep=5pt,
               autogobble,
               %frame=lines,
               fontsize=\footnotesize,
               bgcolor=bg,
               framesep=2mm]{Haskell}
expr2 >>= \_ ->
\end{minted}
\noindent{}所有的do block必须以一个monadic的表达式结尾,而且一个do block的开头是可以用\texttt{let} clause的(但
用在do block中的\texttt{let} clauses不使用\texttt{in}关键字)。上面的\texttt{mothersPaternalGrandfather}可以从
do记法的写法被转换为标准的monadic binding的写法:
\begin{minted}[mathescape=true,
               %linenos,
               numbersep=5pt,
               autogobble,
               %frame=lines,
               fontsize=\footnotesize,
               bgcolor=bg,
               framesep=2mm]{Haskell}
mothersPaternalGrandfather s  =  mother s >>= (\m -> father m >>= (\gf -> father gf))
\end{minted}

\vspace{-0.5em}
\subsubsection{总结}
\indent{} Haskell提供了monads的内建支持。为了利用好这种支持,你要将monad类型构造器声明为\texttt{Monad}类的一个实例,
并提供相应的\texttt{return}与\texttt{>>=}函数的定义。作为\texttt{Monad}类的实例,一个monad可以和do记法一起使用,而do
记法是提供了一种简单而具有命令式风格的记法的语法糖。
\clearpage

\subsection{Monad 定律}
\indent{}到目前为止,本文都避免了技术性讨论,但有几个关于monad的技术点是必须要提及的。Monadic的操作必须
遵从一组定律,即“monad公理”。Haskell编译器并不对这些定律做强制性要求,因而,是否要保证所声明的
\texttt{Monad}实例满足这些定律,是取决于程序员的。Haskell的\texttt{Monad}还包含了一些超出
最小完备定义的函数(目前还未提及)。许多monads遵从标准monad定律之外的附加定律,并且有一个额外的Haskell类
来支持这些扩展的monads。

\subsubsection{三个基本定律}
\indent{}Monad的概念来自范畴论这一数学分支。尽管范畴论的知识对于创建和使用monad并不是必须的,我们确实还是要
遵从一些数学形式。为了创建一个monad,仅仅用正确的类型签名来声明\texttt{Monad}的一个实例是不够的。要使其成为
正确的monad,\texttt{return}和\texttt{>>=}函数必须满足下述三条定律:
\begin{enumerate}
\item 左单位元(left-identity):\texttt{(return x) >>= f  ==  f x}
\item 右单位元(right-identity):\texttt{m >>= return  ==  m}
\item 结合律(associativity):\texttt{(m >>= f) >>= g  ==  m >>= (\textbackslash x -> f x >>= g)}
\end{enumerate}
如上,\noindent{}第一条定律要求\texttt{return}是关于\texttt{>>=}的左单位元。第二条定律要求\texttt{return}是关于
\texttt{>>=}的右单位元。第三条定律是\texttt{>>=}的结合律。遵从这三条定律能够保证使用monad的do记法的语义一致性。

\subsubsection{Failure IS an option}
\indent{}我们在~\ref{subsec:minimal_definition}~中给出的\texttt{Monad}的定义实际上仅是最小完备定义。
\texttt{Monad}类的完全(full)定义实际上还包含两个额外的函数:\texttt{fail}和\texttt{>>}。

\indent{}\texttt{fail}函数的默认实现是:
\begin{minted}[mathescape=true,
               %linenos,
               numbersep=5pt,
               autogobble,
               %frame=lines,
               fontsize=\footnotesize,
               bgcolor=bg,
               framesep=2mm]{Haskell}
fail s  =  error s
\end{minted}
\noindent{}除非你想为failure提供不同的行为,或是将failure融入你的monad策略中,上述实现是不需要被改变的。
关于这个函数,以\texttt{Maybe}举例来说,其\texttt{fail}定义为:
\begin{minted}[mathescape=true,
               %linenos,
               numbersep=5pt,
               autogobble,
               %frame=lines,
               fontsize=\footnotesize,
               bgcolor=bg,
               framesep=2mm]{Haskell}
fail _  =  Nothing
\end{minted}
\noindent{}因而,当与\texttt{Maybe} monad中的其它函数绑定时,\texttt{fail}会返回一个
具有有意义行为的\texttt{Maybe} 实例。

\indent{}\texttt{fail}函数并不是monad的数学定义所需要的部分,但它仍被包含进了\texttt{Monad}的完全定义中,因为
它在Haskell的do记法中起到了作用。在do block中,无论何时一个模式匹配(pattern matching)失败发生,\texttt{fail}函数
就会被调用。下面是这种情形的一个例子:
\begin{minted}[mathescape=true,
               %linenos,
               numbersep=5pt,
               autogobble,
               %frame=lines,
               fontsize=\footnotesize,
               bgcolor=bg,
               framesep=2mm]{Haskell}
fn      :: Int -> Maybe [Int]
fn idx  =  do let l = [Just [1,2,3], Nothing, Just [], Just [7..20]]
              (x:xs) <- l!!idx   -- a pattern match failure will call "fail"
              return xs
\end{minted}
\noindent{}在上面的代码中,\texttt{fn 0}会返回\texttt{Just [2,3]},而\texttt{fn 1}或\texttt{fn 2}都会
得到\texttt{Nothing}。

\indent{}另一个额外的函数\texttt{>>}是一种方便的操作,它被用于bind一个不需要从(计算序列中)之前的计算拿到
其输入的monadic计算。它的定义是关于\texttt{>>=}的:
\begin{minted}[mathescape=true,
               %linenos,
               numbersep=5pt,
               autogobble,
               %frame=lines,
               fontsize=\footnotesize,
               bgcolor=bg,
               framesep=2mm]{Haskell}
(>>)    :: m a -> m b -> m b
m >> k  =  m >>= (\_ -> k)
\end{minted}

\vspace{-1.2em}
\subsubsection{No way out}
\indent{}你可能已经注意到了,在标准的\texttt{Monad}类的定义中,没有对应于从monad中拿到值的方法。实际上,这并不是一个意外。
没有什么能够阻止monad的作者使用特定于对应monad的函数来做到这一点。例如,对于我们熟知的\texttt{Maybe} monad而言,
可以通过在\texttt{Just x}上进行模式匹配,或是使用\texttt{fromJust}函数来从中提取值。

\indent{}因为不需要这种函数,Haskell的\texttt{Monad}类允许单向(one-way)monads的创建。单向monads允许值通过
\texttt{return}函数(有时是\texttt{fail}函数)进入monad,它们允许在monad中进行的计算使用\texttt{>>=}和\texttt{>>}函数,
但它们不允许值从monad中退回。

\indent{}\texttt{IO} monad就是Haskell中单向monad的一个著名的例子。由于你不能从\texttt{IO} monad中逃出,所以不可能写出
一个在\texttt{IO} monad中做计算却不在返回值类型中包含\texttt{IO} 类型构造器的函数。这意味着,对于任何返回值类型中不包含
\texttt{IO}构造器的函数,它肯定不会用到\texttt{IO} monad。对于其它的monads,比如\texttt{List}或\texttt{Maybe},它们确实允许
值从其中逃出,因而你可以写出一个在内部使用了这些monad,却返回non-monadic值的函数。单向monad的妙处在于,它可在其monadic操作
中支持副作用而不使这些副作用破坏程序中non-monadic部分的函数式特性。

\indent{}考虑从用户那里读一个字符这样的简单问题。我们不能简单地去写\texttt{readChar :: Char}这样的函数,因为这种函数每次
被调用时都需要返回一个不同的字符(取决于用户输入了什么)。Haskell作为纯函数式语言的基本特性是所有的函数在形式上都是pure的,
即以相同的参数调用一个函数两次(实际上,无论几次),一定会得到相同的结果。但是我们用\texttt{IO} monad写一个
\texttt{getChar :: IO Char}这样的I/O函数是可以的,因为它只能用于单向monad下的一个序列(sequence)。对于在函数签名中使用
了\texttt{IO}类型构造器的任何函数,是没有办法甩掉\texttt{IO}类型构造器的。这样一来,\texttt{IO}类型构造器就起到了类似
标签(tag)一样的作用,识别所有做I/O的函数。并且,这类函数只在\texttt{IO} monad中有用。因而一个单向monad有效地创建了
一个隔离的计算域,纯函数式语言的规则在这个计算域中可以被放松(relaxed)。函数式的计算可以被移入这个域中,但那些危险
的副作用和非引用透明(non-referentially-transparent)的函数是不能从域中逃出的。

\vspace{-0.6em}
\subsubsection{Zero and Plus}
\indent{}除前文提到的三个基本的monad定律外,一些monads还遵从一些额外的定律。这些monads具有一个特殊的值\texttt{mzero}和一个
运算符\texttt{mplus},它们遵从四个额外的定律:
\begin{enumerate}
\item \texttt{mzero >>= f  ==  mzero}
\item \texttt{m >>= (\textbackslash x -> mzero)  ==  mzero}
\item \texttt{mzero `mplus` m  ==  m}
\item \texttt{m `mplus` mzero  ==  m}
\end{enumerate}
\noindent{}以普通算数运算做类比,如果将\texttt{mzero}看作0,将\texttt{mplus}看作$+$,将\texttt{>>=}看
作$\times$,这是很好理解的。\clearpage

\noindent{}在Haskell中,具有“零”和“加”的monads可被声明为\texttt{MonadPlus}类的实例,它定义如下:
\begin{minted}[mathescape=true,
               %linenos,
               numbersep=5pt,
               autogobble,
               %frame=lines,
               fontsize=\footnotesize,
               bgcolor=bg,
               framesep=2mm]{Haskell}
class (Monad m) => MonadPlus m where
    mzero :: m a
    mplus :: m a -> m a -> m a
\end{minted}
\noindent{}仍用\texttt{Maybe} monad作例子,我们看到它是\texttt{MonadPlus}的实例:
\begin{minted}[mathescape=true,
               %linenos,
               numbersep=5pt,
               autogobble,
               %frame=lines,
               fontsize=\footnotesize,
               bgcolor=bg,
               framesep=2mm]{Haskell}
instance MonadPlus Maybe where
    mzero              =  Nothing
    Nothing `mplus` x  =  x
    x `mplus` _        =  x
\end{minted}
\noindent{}这会将\texttt{Nothing}识别为零值,而且将两个\texttt{Maybe}值加在一起会给出第一个
不为零值(即\texttt{Nothing})的值。如果两个输入值都是零值,那么\texttt{mplus}的结果也是零值。

\indent{}\texttt{mplus}运算符用于将单独(separate)计算的值结合在一起形成一个单一的monadic值。在我们克隆羊例子的
上下文中,我们可以使用\texttt{Maybe}的\texttt{mplus}来定义一个函数:
\begin{minted}[mathescape=true,
               %linenos,
               numbersep=5pt,
               autogobble,
               %frame=lines,
               fontsize=\footnotesize,
               bgcolor=bg,
               framesep=2mm]{Haskell}
parent s  =  (mother s) `mplus` (father s)
\end{minted}
\noindent{}这个函数会返回双亲(若某一方存在的话)。如果没有双亲的任何一方,则会得到\texttt{Nothing}。如果双亲都存在,
这个函数会返回某一方,取决于\texttt{mplus}在\texttt{Maybe} monad中的具体定义。

\subsubsection{总结}
\indent{}\texttt{Monad}类的实例应该满足所谓的monad定律,这些定律描述了monads的代数性质。其中有三个基本定律,表明
\texttt{return}函数同时作为左单位元与右单位元,并且binding运算符是具有结合性的。如果不能满足这些定律,可能会导致
monad行为不正确,在使用do记法时可能会有微妙的问题。

\indent{}除了\texttt{return}和\texttt{>>=}函数,\texttt{Monad}类还定义了另外一个函数\texttt{fail}。该
函数从作为monad的技术需求上来讲不是必须的,但它在实践中很有用,且由于do记法之故,它也被包含在\texttt{Monad}类中。

\indent{}一些monads还遵从额外的monad定律。这样的monad具有零元的概念和加法运算符。Haskell为这些monads提供
了\texttt{MonadPlus}类,它定义了\texttt{mzero}值和\texttt{mplus}运算符。

\indent{}Haskell 2010 standard prelude 包含了\texttt{Monad}类的定义,并提供了一些与monadic数据类型一起使用的辅助函数。
由于篇幅原因,原文介绍的\texttt{sequence, mapM, (=<<), foldM, filterM, zipWithM, when, unless, liftM, ap}等函数的细节
这里就不再详细给出了。

\clearpage

\section{标准Monads的一份目录}
\indent{}这是本文的第二部分。该部分涉及的monads包括标准Haskell的monad与Andy Gill的monad模板库中的一些monad类。Monad模板库
被包含在Glasgow Haskell Compiler(GHC)的\texttt{Control.Monad}下的库中。除了这里涉及的monads,还有很多出现在Haskell的
许多其它地方,比如\texttt{Parsec}中。这些内容超出了我们的讨论范围。本章涉及的monads见下表。
\begin{table}[h]
\resizebox{\textwidth}{!}{
\begin{tabular}{|l|l|l|}
\hline
\textbf{Monad} & \textbf{计算的类型} & \textbf{\texttt{>>=}的组合策略} \\ \hline\hline
\texttt{Identity} & N/A,与monad变换器一起使用 & bound function被应用于输入值 \\ \hline
\texttt{Maybe} & 可能不会返回结果的计算 & \begin{tabular}[c]{@{}l@{}}\texttt{Nothing}输入会
给出\texttt{Nothing}输出\\ \texttt{Just x}输入将\texttt{x}作为bound function的输入\end{tabular} \\ \hline
\texttt{Error} & 可能会失败或抛异常的计算 &  binding在不执行bound function的情况下传递失败信息,或将
成功值作为bound function的输入\\ \hline
\texttt{[]}(\texttt{List}) & 会返回多个可能值的非确定性计算 & 将bound function映射到输入
列表上,然后将结果列表连起来以得到一个从所有可能输入产生的所有可能结果构成的列表 \\ \hline
\texttt{IO} & 执行I/O的计算 & 按binding序的I/O动作的顺序执行 \\ \hline
\texttt{State} & 维护状态的计算 & bound function被应用到输入值上以产生一个要被应用到输入状态上的状态转移函数 \\ \hline
\texttt{Reader} & 从共享环境中读的计算 & bound function被应用到使用相同环境的输入值上 \\ \hline
\texttt{Writer} & 除计算值外还写数据的计算 & 写数据分离于值被维护,bound function被应用于输入
值,它写的任何东西都会被追加到写入数据流中 \\ \hline
\texttt{Cont} & 可被中断和重启的计算 & bound function被插入延续链中 \\ \hline
\end{tabular}}
\end{table}

\subsection{The Identity monad}
\begin{itemize}[leftmargin=*,topsep=0pt,itemsep=0pt]
\item \textbf{计算的类型:}简单的函数应用
\item \textbf{Binding策略:}bound function被应用于输入值,\texttt{Identity x >>= f == f x}
\item \textbf{适用于:}将monad变换器应用于Identity monad可衍生monads
\item \textbf{zero 与 plus:}无
\item \textbf{样例类型:}\texttt{Identity a}
\end{itemize}

\subsubsection{动机}
\indent{}Identity monad是一个不体现任何计算策略的monad,它简单地将bound function应用到其输入上。从计算上来说,
没有任何理由不直接使用更为简单直接的函数应用,却去用Identity monad。但Identity monad的目的是作为monad变换器相关
理论的基础。任何应用到Identity monad上的monad变换器都会给出一个对应monad的非变换器(non-transformer)版本。

\subsubsection{定义}
\vspace{-1em}
\begin{minted}[mathescape=true,
               %linenos,
               numbersep=5pt,
               autogobble,
               frame=lines,
               fontsize=\footnotesize,
               bgcolor=bg,
               framesep=2mm]{Haskell}
newtype Identity a  =  Identity { runIdentity :: a }

instance Monad Identity where
    return a            =  Identity a   -- i.e. return = id
    (Identity x) >>= f  =  f x          -- i.e. x >>= f = f x
\end{minted}
\noindent{}在类型定义中用到了\texttt{runIdentity}标签,因为它遵从一种monad定义的样式,这种样式将monad值
显式地表示为计算。由于Identity monad不做任何计算,它的定义是trivial的。
\begin{minted}[mathescape=true,
               %linenos,
               numbersep=5pt,
               autogobble,
               %frame=lines,
               fontsize=\footnotesize,
               bgcolor=bg,
               framesep=2mm]{Haskell}
type State s a  =  StateT s Identity a
\end{minted}
如上所示,Identity monad的一个典型的用途是从一个monad变换器导出monad。
\clearpage

\subsection{The Maybe monad}
\begin{itemize}[leftmargin=*,topsep=0pt,itemsep=0pt]
\item \textbf{计算的类型:}可能会返回\texttt{Nothing}的计算
\item \textbf{Binding策略:}\texttt{Nothing}值会绕过bound function,而其它值会作为bound function的输入
\item \textbf{适用于:}从一列可能会返回\texttt{Nothing}的函数建立计算,比如一系列字典查找请求
\item \textbf{zero 与 plus:}\texttt{Nothing}是zero;若两个输入不都是\texttt{Nothing},则plus给出第一个非\texttt{Nothing}值
\item \textbf{样例类型:}\texttt{Maybe a}
\end{itemize}

\subsubsection{动机}
\indent{}Maybe monad体现了这样一种策略,它将一连串的计算组合起来,这些计算的每一个都可能会给出\texttt{Nothing},若
其中任何一步给出了\texttt{Nothing},则提前结束并以\texttt{Nothing}返回。当计算需要一连串相互依赖的步骤,且步骤可能
无法返回值时,这是特别有用的。

\vspace{-0.5em}
\subsubsection{定义}
\vspace{-1em}
\begin{minted}[mathescape=true,
               %linenos,
               numbersep=5pt,
               autogobble,
               frame=lines,
               fontsize=\footnotesize,
               bgcolor=bg,
               framesep=2mm]{Haskell}
data  Maybe a  =  Nothing | Just a

instance Monad Maybe where
    return          =  Just
    fail            =  Nothing
    Nothing  >>= f  =  Nothing
    (Just x) >>= f  =  f x

instance MonadPlus Maybe where
    mzero              =  Nothing
    Nothing `mplus` x  =  x
    x `mplus` _        =  x
\end{minted}

\vspace{-0.5em}
\subsubsection{例子}
\indent{}一般性的例子是结合字典查询。给定一个将完整名字映射到邮箱地址的字典,和一个将昵称映射到邮箱地址的字典,
再给出一个将邮箱地址映射到邮箱首选项的字典,那么可以创建一个根据全名或昵称来寻找对应用户邮箱首选项的函数:
\begin{minted}[mathescape=true,
               linenos,
               numbersep=5pt,
               autogobble,
               frame=lines,
               fontsize=\footnotesize,
               bgcolor=bg,
               framesep=2mm]{Haskell}
data MailPref = HTML | Plain
data MailSystem = ...

getMailPrefs :: MailSystem -> String -> Maybe MailPref
getMailPrefs sys name =
  do let nameDB = fullNameDB sys
         nickDB = nickNameDB sys
         prefDB = prefsDB sys
  addr <- (lookup name nameDB) `mplus` (lookup name nickDB)
  lookup addr prefDB
\end{minted}
\clearpage

\subsection{The Error monad}
\begin{itemize}[leftmargin=*,topsep=0pt,itemsep=0pt]
\item \textbf{计算的类型:}可能会失败或抛异常的计算
\item \textbf{Binding策略:}失败会记录其原因或位置,失败值会绕过bound function,而其它值会作为其输入
\item \textbf{适用于:}从可能失败的函数序列中构建计算,或者使用异常处理来构建错误处理
\item \textbf{zero 与 plus:}zero由空错误表示;如果第一个参数失败,plus会执行其第二个参数
\item \textbf{样例类型:}\texttt{Either String a}
\end{itemize}

\subsubsection{动机}
\indent{}Error monad(也叫Exception monad)体现了这样一种策略,它通过绕过从异常抛出点到异常处理点的bound function来
组合可能会抛异常的计算。

\indent{}\texttt{MonadError}类在错误信息类型与monad类型构造器上参数化。对于以字符串形式作为错误描述的error monad,通常
使用\texttt{Either String}作monad类型构造器。在这种情况(还有许多通常情况)下,得到的monad已经被定义
为\texttt{MonadError}类的实例了。你也可以定义自己的错误类型或是使用除\texttt{Either String}与\texttt{Either IOError}
外的monad类型构造器。在这些情况下,你就必须要显式地去定义\texttt{Error}及\texttt{MonadError}类的实例了。

\subsubsection{定义}
\indent{}\texttt{MonadError}类的定义如下。它使用了多参数类型类和funDeps(它们都是超出了Haskell 2010的
语言扩展),但无需了解它们即可使用\texttt{MonadError}。
\begin{minted}[mathescape=true,
               %linenos,
               numbersep=5pt,
               autogobble,
               frame=lines,
               fontsize=\footnotesize,
               bgcolor=bg,
               framesep=2mm]{Haskell}
class Error a where
    noMsg  :: a
    strMsg :: String -> a

class (Monad m) => MonadError e m | m -> e where
    throwError :: e -> m a
    catchError :: m a -> (e -> m a) -> m a
\end{minted}
\noindent{}\texttt{throwError}在一个monadic的计算中使用,以开始异常处理。\texttt{catchError}提供了
用来解决之前的错误的handler函数,并返回到常规执行中。一个常见的用法是:
\begin{minted}[mathescape=true,
               %linenos,
               numbersep=5pt,
               autogobble,
               %frame=lines,
               fontsize=\footnotesize,
               bgcolor=bg,
               framesep=2mm]{Haskell}
do { action1; action2; action3 } `catchError` handler
\end{minted}
\noindent{}这些\texttt{action}函数会调用\texttt{throwError}。注意\texttt{handler}和do block必须有相同的返回类型。

\indent{}将\texttt{Either a}类型构造器定义为\texttt{MonadError}的实例是很直接的。遵循惯例,\texttt{Left}用于错误值,
而\texttt{Right}用于非错误(non-error)值。
\begin{minted}[mathescape=true,
               %linenos,
               numbersep=5pt,
               autogobble,
               frame=lines,
               fontsize=\footnotesize,
               bgcolor=bg,
               framesep=2mm]{Haskell}
instance MonadError (Either e) where
    throwError                     =  Left
    (Left e) `catchError` handler  =  handler e
    a        `catchError` _        =  a
\end{minted}

\subsubsection{例子}
\indent{}接下来我们看一个例子。这个例子演示了如何将自定义的\texttt{Error}数据类型与\texttt{ErrorMonad}的
\texttt{throwError}与\texttt{catchError}异常处理机制一起使用。这个例子尝试解析十六进制数,如果遇到了非法字符
会抛异常。我们使用自定义的\texttt{Error}数据类型来记录解析错误发生的位置。异常被一个函数调用捕获,并通过打印
错误信息的方式来处理。
\vspace{-0.8em}
\begin{minted}[mathescape=true,
               linenos,
               numbersep=5pt,
               autogobble,
               frame=lines,
               fontsize=\footnotesize,
               bgcolor=bg,
               framesep=2mm]{Haskell}
-- This is the type of our parse error representation.
data ParseError = Err {location::Int, reason::String}

-- We make it an instance of the Error class
instance Error ParseError where
  noMsg    = Err 0 "Parse Error"
  strMsg s = Err 0 s

-- For our monad type constructor, we use Either ParseError
-- which represents failure using Left ParseError or a
-- successful result of type a using Right a.
type ParseMonad = Either ParseError

-- parseHexDigit attempts to convert a single hex digit into
-- an Integer in the ParseMonad monad and throws an error on an invalid character
parseHexDigit :: Char -> Int -> ParseMonad Integer
parseHexDigit c idx = if isHexDigit c then
                        return (toInteger (digitToInt c))
                      else
                        throwError (Err idx ("Invalid character '" ++ [c] ++ "'"))

-- parseHex parses a string containing a hexadecimal number into
-- an Integer in the ParseMonad monad.  A parse error from parseHexDigit
-- will cause an exceptional return from parseHex.
parseHex :: String -> ParseMonad Integer
parseHex s = parseHex' s 0 1
  where parseHex' []      val _   = return val
        parseHex' (c:cs)  val idx = do d <- parseHexDigit c idx
                                       parseHex' cs ((val * 16) + d) (idx + 1)

-- toString converts an Integer into a String in the ParseMonad monad
toString :: Integer -> ParseMonad String
toString n = return $ show n

-- convert takes a String containing a hexadecimal representation of
-- a number to a String containing a decimal representation of that
-- number.  A parse error on the input String will generate a
-- descriptive error message as the output String.
convert :: String -> String
convert s = let (Right str) = do {n <- parseHex s; toString n} `catchError` printError
            in str
  where printError e = return $ "At index " ++ (show (location e)) ++ ":" ++ (reason e)
\end{minted}
\clearpage

\subsection{The List monad}
\begin{itemize}[leftmargin=*,topsep=0pt,itemsep=0pt]
\item \textbf{计算的类型:}可能会返回零个、一个或多个可能结果的计算
\item \textbf{Binding策略:}bound function被应用于输入list的所有可能值上,并连接list结果以产生返回list
\item \textbf{适用于:}从非确定性运算的序列构建计算,比如解析非确定语法
\item \textbf{zero 与 plus:}\texttt{[]}是zero;\texttt{++}是plus操作
\item \textbf{样例类型:}\texttt{[a]}
\end{itemize}

\vspace{-0.5em}
\subsubsection{动机}
\vspace{-0.5em}
\indent{}List monad体现了这样一种策略,它通过将操作应用到每一步的所有可能的值上来组合一连串的非确定性计算。
这种情况下,它会在所有的模糊之处被消除前探索所有的可能性。

\vspace{-0.5em}
\subsubsection{定义}
\vspace{-1.5em}
\begin{minted}[mathescape=true,
               %linenos,
               numbersep=5pt,
               autogobble,
               frame=lines,
               fontsize=\footnotesize,
               bgcolor=bg,
               framesep=2mm]{Haskell}
instance Monad [] where
    m >>= f   =  concatMap f m
    return x  =  [x]
    fail s    =  []

instance MonadPlus [] where
    mzero  =  []
    mplus  =  (++)
\end{minted}

\vspace{-1em}
\subsubsection{例子}
\indent{}这里我们以一个简单的情景问题作为例子,更复杂的例子可参见原文。假设我们想找出扑克牌的所有可能的模式,
我们知道一副扑克牌的数字是从1(Ace)到13(King),并且每张牌可能的装饰有梅花、钻石、心形和桃形。如果不考虑
颜色,这些模式可以由这样的方式给出:
\begin{minted}[mathescape=true,
               linenos,
               numbersep=5pt,
               autogobble,
               frame=lines,
               fontsize=\footnotesize,
               bgcolor=bg,
               framesep=2mm]{Haskell}
-- 4 categories of suit
data Suit  =  Club | Diamond | Heart | Spade deriving (Show, Enum)

-- rank range in 1 ~ 13
data Rank  =  Rank Int

-- names of rank
instance Show Rank where
    show (Rank 1)  = "Ace"
    show (Rank 11) = "Jack"
    show (Rank 12) = "Queen"
    show (Rank 13) = "King"
    show (Rank i)  = show i

-- cards in a deck
deck = [(Rank r, s) | s <- [Club .. Spade]
                    , r <- [1..13]]
\end{minted}
\clearpage

\subsection{The IO monad}
\begin{itemize}[leftmargin=*,topsep=0pt,itemsep=0pt]
\item \textbf{计算的类型:}执行I/O的计算
\item \textbf{Binding策略:}I/O动作以其被绑定的顺序执行,若失败会抛I/O错误,错误可被捕获和处理
\item \textbf{适用于:}在Haskell程序中做I/O
\item \textbf{zero 与 plus:}无
\item \textbf{样例类型:}\texttt{IO a}
\end{itemize}

\vspace{-0.5em}
\subsubsection{动机}
\vspace{-0.5em}
\indent{}由于输入/输出没有引用透明性且带有副作用,它们和纯函数式语言是不相容的。IO monad通过将做I/O的计算
限制在IO monad里来解决该问题。

\vspace{-0.5em}
\subsubsection{定义}
\vspace{-0.5em}
\indent{}IO monad的定义是与平台相关的(platform-specific)。没有可以用于将数据从IO monad中移出的数据构造器或函数
被提供,这使得IO monad是一个单向monad,而这一点是保证函数式程序安全性的基础。它将IO monad中的命令式风格的
副作用或非引用透明的动作隔离起来。

\indent{}在整篇文章中,我们都将monadic值称为计算。然而,IO monad中的值通常被称为I/O动作(I/O actions),我们会在
这里使用这个更为贴切的术语。

\indent{}在Haskell中,顶层(top-level)的\texttt{main}函数的类型必须是\texttt{IO ()},这样一来程序通常在顶层被构建
为一个命令式风格的I/O动作序列和对函数式风格代码的调用。从\texttt{IO}模块导出的函数本身并不做I/O,而是返回I/O动作,
这些I/O动作描述了要执行的I/O操作。I/O动作可以在IO monad中合并(以纯函数式的方式)以创建更为复杂的I/O动作,则
最终的I/O动作即为程序的\texttt{main}值。

\indent{}标准prelude和\texttt{IO}模块定义了很多可以用于IO monad中的函数,任何Haskell程序员无疑都会熟悉其中的
一些。这里,我们将仅讨论IO monad的monadic的方面,而非所有的那些函数。

\indent{}\texttt{IO}类型构造器是\texttt{Monad}类与\texttt{MonadError}类的实例,其中error的类型是\texttt{IOError}。
\texttt{fail}被定义为抛出一个以字符串参数建立的error。如果导入(import)了\texttt{Control.Monad.Error}模块,就
可以在\texttt{IO} monad中使用其中的异常机制。同样的机制还有\texttt{IO}模块导出的其他名称:
\texttt{ioError}和\texttt{catch}。
\begin{minted}[mathescape=true,
               %linenos,
               numbersep=5pt,
               autogobble,
               frame=lines,
               fontsize=\footnotesize,
               bgcolor=bg,
               framesep=2mm]{Haskell}
instance Monad IO where
    return a  =  ...   -- function from a -> IO a
    m >>= k   =  ...   -- executes the I/O action m and binds the value to k's input
    fail s    =  ioError (userError s)

data  IOError = ...

ioError :: IOError -> IO a
ioError = ...

userError :: String -> IOError
userError = ...

catch :: IO a -> (IOError -> IO a) -> IO a
catch = ...

try :: IO a -> IO (Either IOError a)
try f = catch (do r <- f
                  return (Right r))
              (return . Left)
\end{minted}

\noindent{}\texttt{IO} monad作为\texttt{MonadError}的实例被纳入Monad模板库中:
\vspace{-0.2em}
\begin{minted}[mathescape=true,
               %linenos,
               numbersep=5pt,
               autogobble,
               frame=lines,
               fontsize=\footnotesize,
               bgcolor=bg,
               framesep=2mm]{Haskell}
instance Error IOError where
  ...

instance MonadError IO where
    throwError = ioError
    catchError = catch
\end{minted}
\noindent{}\texttt{IO}模块导出了一个名为\texttt{try}的便利的函数,它执行I/O动作,且能在一个I/O错误被捕获时
返回\texttt{Left IOError},或在动作成功时返回\texttt{Right result}。

\vspace{-0.5em}
\subsubsection{例子}
\vspace{-0.5em}
\indent{}下面的例子展示了命令\texttt{tr}的部分实现,这个命令将将标准输入流拷贝到标准输出流,字符转换由命令行
参数控制。它展示了与\texttt{IO} monad协同使用的\texttt{MonadError}的异常处理机制。
\vspace{-0.5em}
\begin{minted}[mathescape=true,
               linenos,
               numbersep=5pt,
               autogobble,
               frame=lines,
               fontsize=\footnotesize,
               bgcolor=bg,
               framesep=2mm]{Haskell}
import Monad
import System
import IO
import Control.Monad.Error

-- translate char in set1 to corresponding char in set2
translate :: String -> String -> Char -> Char
translate []     _      c  =  c
translate (x:xs) []     c  =  if x == c then ' ' else translate xs []  c
translate (x:xs) [y]    c  =  if x == c then  y  else translate xs [y] c
translate (x:xs) (y:ys) c  =  if x == c then  y  else translate xs ys  c

-- translate an entire string
translateString :: String -> String -> String -> String
translateString set1 set2 str = map (translate set1 set2) str

usage :: IOError -> IO ()
usage e = do putStrLn "Usage: ex14 set1 set2"
             putStrLn "Translates characters in set1 on stdin to the corresponding"
             putStrLn "characters from set2 and writes the translation to stdout."

-- translates stdin to stdout based on commandline arguments
main :: IO ()
main = (do [set1, set2] <- getArgs
           contents     <- hGetContents stdin
           putStr $ translateString set1 set2 contents) `catchError` usage
\end{minted}
\clearpage

\subsection{The State monad}
\begin{itemize}[leftmargin=*,topsep=0pt,itemsep=0pt]
\item \textbf{计算的类型:}维护状态的计算
\item \textbf{Binding策略:}将一个状态参数贯穿于bound function序列中,因而同一状态不会被使用两次,这
                            $\textrm{ }\quad\quad\quad\quad\quad\;\;\quad$样会给人一种原地更新的错觉
\item \textbf{适用于:}从需要共享状态的操作序列中构建计算
\item \textbf{zero 与 plus:}无
\item \textbf{样例类型:}\texttt{State st a}
\end{itemize}

\vspace{-1em}
\subsubsection{动机}
\vspace{-0.5em}
\indent{}纯函数式语言不能原地更新值,因为这会打破引用透明性。模拟这种带状态的计算的惯用法是将一个状态参数
“贯穿(thread)”于一个函数序列中:
\vspace{-0.8em}
\begin{minted}[mathescape=true,
               %linenos,
               numbersep=5pt,
               autogobble,
               frame=lines,
               fontsize=\footnotesize,
               bgcolor=bg,
               framesep=2mm]{Haskell}
data MyType  =  MT Int Bool Char Int deriving Show

makeRandomValue :: StdGen -> (MyType, StdGen)
makeRandomValue g  =  let (n,g1) = randomR (1,100)   g
                          (b,g2) = random  g1
                          (c,g3) = randomR ('a','z') g2
                          (m,g4) = randomR (-n,n)    g3
                      in (MT n b c m, g4)
\end{minted}
\noindent{}这样做能起作用,但是这样的代码很容易出错,很乱而且难以维护。State monad将状态参数的threading隐
藏在binding操作中,同时也使代码更易编写,更容易读,也更容易修改。

\vspace{-0.8em}
\subsubsection{定义}
\vspace{-1.5em}
\begin{minted}[mathescape=true,
               %linenos,
               numbersep=5pt,
               autogobble,
               frame=lines,
               fontsize=\footnotesize,
               bgcolor=bg,
               framesep=2mm]{Haskell}
newtype State s a  =  State { runState :: (s -> (a,s)) }

instance Monad (State s) where
    return a         =  State $ \s -> (a,s)
    (State x) >>= f  =  State $ \s -> let (v,s') = x s in runState (f v) s'
\end{minted}
\noindent{}State monad中的值被表示为从一个初始状态到\texttt{(value, newState)}对的transition function,并且提供了一个
newtype 定义:\texttt{State s a},它是在带有类型为\texttt{s}的状态的state monad中类型为\texttt{a}的值的类型。
\texttt{MonadState}类为state monad提供了一个标准而简单的接口:
\vspace{-0.5em}
\begin{minted}[mathescape=true,
               %linenos,
               numbersep=5pt,
               autogobble,
               frame=lines,
               fontsize=\footnotesize,
               bgcolor=bg,
               framesep=2mm]{Haskell}
class MonadState m s | m -> s where
    get :: m s
    put :: s -> m ()

instance MonadState (State s) s where
    get    =  State $ \s -> (s,s)
    put s  =  State $ \_ -> ((),s)
\end{minted}
\noindent{}\texttt{get}函数通过将状态拷贝并作为value对其进行检索。\texttt{put}设置monad状态但不给出值。

\indent{}还有许多基于\texttt{get}和\texttt{put}构建的额外的有用函数,这些函数进行更为复杂的计算。这里不详细
展开了,具体内容可以参见Haskell的state monad库。

\subsubsection{例子}
\indent{}先来熟悉一下本节提到的几个函数的简单用法:
\begin{minted}[mathescape=true,
               linenos,
               numbersep=5pt,
               autogobble,
               frame=lines,
               fontsize=\footnotesize,
               bgcolor=bg,
               framesep=2mm]{Haskell}
-- "return" set the result value but leave the state unchanged
runState (return 'X') 1  -- => ('X', 1)

-- "get" set the result value to the state and leave the state unchanged
runState get 1  -- => (1, 1)

-- "put" set the result value to () and set the state value
runState (put 5) 1  -- => ((), 5)
\end{minted}

\indent{}State monad的一个简单应用是将随机数生成器的状态贯穿于对生成函数的多次调用中。
\begin{minted}[mathescape=true,
               linenos,
               numbersep=5pt,
               autogobble,
               frame=lines,
               fontsize=\footnotesize,
               bgcolor=bg,
               framesep=2mm]{Haskell}
data MyType = MT Int Bool Char Int deriving Show

{- Using the State monad, we can define a function that returns
   a random value and updates the random generator state at
   the same time.
-}
getAny :: (Random a) => State StdGen a
getAny = do g      <- get
            (x,g') <- return $ random g
            put g'
            return x

-- similar to getAny, but it bounds the random value returned
getOne :: (Random a) => (a,a) -> State StdGen a
getOne bounds = do g      <- get
                   (x,g') <- return $ randomR bounds g
                   put g'
                   return x

{- Using the State monad with StdGen as the state, we can build
   random complex types without manually threading the
   random generator states through the code.
-}
makeRandomValueST :: StdGen -> (MyType, StdGen)
makeRandomValueST = runState (do n <- getOne (1,100)
                                 b <- getAny
                                 c <- getOne ('a','z')
                                 m <- getOne (-n,n)
                                 return (MT n b c m))
\end{minted}
\clearpage

\subsection{The Reader monad}
\begin{itemize}[leftmargin=*,topsep=0pt,itemsep=0pt]
\item \textbf{计算的类型:}从共享环境中读值的计算
\item \textbf{Binding策略:}Monad值是从环境到值的函数,bound function与bound value都可访问共享环境
\item \textbf{适用于:}维护变量绑定,或用于其它共享环境
\item \textbf{zero 与 plus:}无
\item \textbf{样例类型:}\texttt{Reader [(String, Value)] a}
\end{itemize}

\subsubsection{动机}
\indent{}一些编程问题需要在共享环境(或一个变量绑定集合)中做计算。这些计算需要从共享环境中读值,并且有时候
还需要在修改的环境(绑定被覆盖,或有新的绑定)中执行子计算。
Reader monad就是针对这些类型的计算特殊设计的,比起使用state monad,它们在这些情况下更为清晰和简洁。

\subsubsection{定义}
\indent{}下面的定义用到了不在Haskell 2010中的语言扩展:多参数类型类与funDeps,但这并不妨碍我们去用好reader monad,尽管
我们可能并不太了解这些扩展。
\begin{minted}[mathescape=true,
               %linenos,
               numbersep=5pt,
               autogobble,
               frame=lines,
               fontsize=\footnotesize,
               bgcolor=bg,
               framesep=2mm]{Haskell}
newtype Reader e a  =  Reader { runReader :: (e -> a) }

instance Monad (Reader e) where
    return a          =  Reader $ \e -> a
    (Reader r) >>= f  =  Reader $ \e -> runReader (f (r e)) e
\end{minted}
\noindent{}Reader monad中的值是从环境到值的函数。为了从reader monad中的计算里提取最终的值,可以简单地
将\texttt{(runReader reader)}应用到环境值上。

\indent{}\texttt{return}函数会创建这样一个\texttt{Reader}:它忽略环境并给出给定的值。而binding操作会创建
这样一个\texttt{Reader}:它使用环境对\texttt{>>=}左侧的值进行提取,再在相同的环境中
将bound function应用到提取到的值上。再来看看\texttt{MonadReader}类:
\begin{minted}[mathescape=true,
               %linenos,
               numbersep=5pt,
               autogobble,
               frame=lines,
               fontsize=\footnotesize,
               bgcolor=bg,
               framesep=2mm]{Haskell}
class MonadReader e m | m -> e where
    ask   :: m e
    local :: (e -> e) -> m a -> m a

instance MonadReader e (Reader e) where
    ask        =  Reader id
    local f c  =  Reader $ \e -> runReader c (f e)

asks :: (MonadReader e m) => (e -> a) -> m a
asks sel  =  ask >>= return . sel
\end{minted}
\noindent{}\texttt{MonadReader}类提供了一些有用的函数。\texttt{ask}函数从环境中检索,而\texttt{local}函数
在一个修改的环境中执行计算。\texttt{asks}检索当前环境的函数,并通常与选择器(selector)或查找函数一起使用。

\subsubsection{例子}
\indent{}考虑模板实例化的问题,这可能包含变量替换与模板引入。使用reader monad,我们可以维护一个环境,这个环境
是有关所有已知模板与所有已知的变量绑定的。之后,当变量替换发生时,我们可以用\texttt{asks}函数来查找变量的值。
而当一个带有新的变量定义的模板被引入时,我们可以使用\texttt{local}函数在包含额外变量绑定的修改后的环境中
解析(resolve)模板。
\vspace{-0.8em}
\begin{minted}[mathescape=true,
               linenos,
               numbersep=5pt,
               autogobble,
               frame=lines,
               fontsize=\footnotesize,
               bgcolor=bg,
               framesep=2mm]{Haskell}
-- This the abstract syntax representation of a template
--                  Text       Variable     Quote        Include                   Compound
data Template    =  T String | V Template | Q Template | I Template [Definition] | C [Template]
data Definition  =  D Template Template

-- Our environment consists of two association lists
data Environment = Env {templates::[(String,Template)], variables::[(String,String)]}

-- lookup a variable from the environment
lookupVar :: String -> Environment -> Maybe String
lookupVar name env  =  lookup name (variables env)

-- lookup a template from the environment
lookupTemplate :: String -> Environment -> Maybe Template
lookupTemplate name env  =  lookup name (templates env)

-- add a list of resolved definitions to the environment
addDefs :: [(String,String)] -> Environment -> Environment
addDefs defs env  =  env {variables = defs ++ (variables env)}

-- resolve a Definition and produce a (name,value) pair
resolveDef :: Definition -> Reader Environment (String,String)
resolveDef (D t d) = do name <- resolve t
                        value <- resolve d
                        return (name,value)

-- resolve a template into a string
resolve :: Template -> Reader Environment (String)
resolve (T s)    = return s
resolve (V t)    = do varName  <- resolve t
                      varValue <- asks (lookupVar varName)
                      return $ maybe "" id varValue
resolve (Q t)    = do tmplName <- resolve t
                      body     <- asks (lookupTemplate tmplName)
                      return $ maybe "" show body
resolve (I t ds) = do tmplName <- resolve t
                      body     <- asks (lookupTemplate tmplName)
                      case body of
                        Just t' -> do defs <- mapM resolveDef ds
                                      local (addDefs defs) (resolve t')
                        Nothing -> return ""
resolve (C ts)   = (liftM concat) (mapM resolve ts)
\end{minted}
\clearpage

\subsection{The Writer monad}
\begin{itemize}[leftmargin=*,topsep=0pt,itemsep=0pt]
\item \textbf{计算的类型:}产生计算值与数据流的计算
\item \textbf{Binding策略:}Writer monad值是\texttt{(computation value, log value)}对。Binding会将计算值替
                            $\textrm{ }\quad\quad\quad\quad\quad\;\;\quad$
                            换为把bound function应用到之前的值上得到的结果,并且将计算中的任何日志数
                            $\textrm{ }\quad\quad\quad\quad\quad\;\;\quad$
                            据追加(append)到现有日志数据中。
\item \textbf{适用于:}记录(logging),或其它产生副产物输出(output ``on the side'')的计算
\item \textbf{zero 与 plus:}无
\item \textbf{样例类型:}\texttt{Writer [String] a}
\end{itemize}

\subsubsection{动机}
\indent{}很多时候我们想要一个产生副产物的输出,日志和追踪(tracing)信息就是最常见的例子。这种作为“副产物”的数据
在执行计算时被生成出来,但并不作为计算的主要结果,我们只是想保留它们。

\indent{}显式地管理日志或追踪数据会使代码变得杂乱,并可能招致一些细微的错误,比如漏日志。Writer monad提供了
一种管理这种输出的更为干净的方式,而不使主计算变得混乱。

\subsubsection{定义}
\indent{}为了能够理解这里的定义,有关monoid(幺半群)的知识是必须要了解的。monoid比monad要简单一些,这里简单介绍
一下:幺半群是一个带有可结合二元运算和单位元的代数结构。一个monoid要遵从一些数学定律:结合律、单位元与封闭性。
你可能会发现这些定律与\texttt{MonadPlus}实例的\texttt{mzero}与\texttt{mplus}上的定律是一样的,因为带有zero与plus的
monad同时也是monoid。带有二元加法和单位元0的自然数,带有二元乘法与单位元1的自然数,这两者都是数学上monoid的经典例子。

\indent{}在Haskell中,一个monoid包括一个类型,一个单位元,还有一个二元运算。Haskell在模组\texttt{Data.Monoid}中定义了
\texttt{Monoid}类,以此提供使用monoids的标准框架:单位元命名为\texttt{mempty},二元操作命名为\texttt{mappend}。Haskell
中最常用的标准monoid包括list和类型为\texttt{(a -> a)}的函数。

\indent{}在将list作为monoid用于Writer时应该小心,因为随着输出规模的增长,可能会出现与\texttt{mappend}操作相关的
性能问题。这种情况下,最好选用支持快速追加操作的数据结构。

\indent{}如下所示,writer monad维护一个\texttt{(value, log)}对。在这个pair中,日志(log)的类型必须是
一个monoid。\texttt{return}函数简单地返回带有空日志的值。Binding操作将当前的值作为bound function的输入,
并将输出日志追加到现有日志中。
\begin{minted}[mathescape=true,
               %linenos,
               numbersep=5pt,
               autogobble,
               frame=lines,
               fontsize=\footnotesize,
               bgcolor=bg,
               framesep=2mm]{Haskell}
newtype Writer w a  =  Writer { runWriter :: (a,w) }

instance (Monoid w) => Monad (Writer w) where
    return a              =  Writer (a,mempty)
    (Writer (a,w)) >>= f  =  let (a',w') = runWriter $ f a
                             in Writer (a',w `mappend` w')
\end{minted}

\indent{}上面的定义用到了不在 Haskell 2010 中的语言扩展:多参数类型类与 funDeps,但这并不妨碍我们用好Writer monad,
尽管我们可能并不太了解这些扩展。
\begin{minted}[mathescape=true,
               %linenos,
               numbersep=5pt,
               autogobble,
               frame=lines,
               fontsize=\footnotesize,
               bgcolor=bg,
               framesep=2mm]{Haskell}
class (Monoid w, Monad m) => MonadWriter w m | m -> w where
    pass   :: m (a, w -> w) -> m a
    listen :: m a -> m (a,w)
    tell   :: w -> m ()

instance (Monoid w) => MonadWriter w (Writer w) where
    pass   (Writer ((a,f),w)) = Writer (a,f w)
    listen (Writer (a,w))     = Writer ((a,w),w)
    tell   s                  = Writer ((),s)

listens :: (MonadWriter w m) => (w -> b) -> m a -> m (a,b)
listens f m = do (a,w) <- listen m
                 return (a,f w)

censor :: (MonadWriter w m) => (w -> w) -> m a -> m a
censor f m = pass $ do a <- m
                       return (a,f)
\end{minted}
\noindent{}最为简单而有用的函数是\texttt{tell},它为日志添加一个或多个条目。\texttt{listen}函数将一个writer转为另一个。
待转的writer返回\texttt{a}并产生输出\texttt{w},它被转为另一个writer,这个writer产生值\texttt{(a,w)}且仍旧产生输出
\texttt{w}。这使得计算可以“监听”writer生成的日志输出。\texttt{pass}函数有点复杂,它也将一个writer转为另一个。待转的writer
产生值\texttt{(a,f)}且产生输出\texttt{w},它被转为一个产生值\texttt{a}而产生输出\texttt{f w}的writer。这略显麻烦,因而
通常会使用辅助函数\texttt{censor}。\texttt{censor}取一个函数和一个writer,并产生一个新的writer,它的输出
与给定的那个writer相同,但其日志条目是被给定的那个函数修改过的。

\indent{}还有一个叫\texttt{listens}(注意,不是\texttt{listen})的函数。除了日志部分要被所提供的函数修改外,
这个函数做的事情和\texttt{listen}是相同的。

\subsubsection{例子}
在这个例子中,让我们想象一个非常简单的防火墙,它会对数据包做过滤。过滤会按照关于源/目的主机与
数据包有效负载的匹配规则来进行。这个防火墙主要的工作是对数据包进行过滤,但我们也想让它产生活动日志,以便
追踪与过滤相关的信息。
\begin{minted}[mathescape=true,
               linenos,
               numbersep=5pt,
               autogobble,
               frame=lines,
               fontsize=\footnotesize,
               bgcolor=bg,
               framesep=2mm]{Haskell}
-- this is the format of our log entries
data Entry = Log {count::Int, msg::String} deriving Eq

-- add a message to the log
logMsg :: String -> Writer [Entry] ()
logMsg s = tell [Log 1 s]

-- this handles one packet
filterOne :: [Rule] -> Packet -> Writer [Entry] (Maybe Packet)
filterOne rules packet = do
        rule <- return (match rules packet)
        case rule of
                Nothing  -> do
                        logMsg $ "DROPPING UNMATCHED PACKET: " ++ (show packet)
                        return Nothing
                (Just r) -> do
                        when (logIt r) $ logMsg ("MATCH: " ++ (show r) ++ " <=> " ++ (show packet))
                        case r of (Rule Accept _ _) -> return $ Just packet
                                  (Rule Reject _ _) -> return Nothing
\end{minted}
\indent{}可以看到,这是很简单的。但是如果我们想要将几条连续重复的日志条目进行合并呢?现有的这些函数没办法
让我们修改前几个阶段的计算过程的输出,但我们可以使用“惰性日志”技巧,只有当我们得到一个新的无重复条目时,才将
日志加入。
\vspace{-0.5em}
\begin{minted}[mathescape=true,
               linenos,
               numbersep=5pt,
               autogobble,
               frame=lines,
               fontsize=\footnotesize,
               bgcolor=bg,
               framesep=2mm]{Haskell}
-- merge identical entries at the end of the log
-- This function uses [Entry] as both the log type and the result type.
-- When two identical messages are merged, the result is just the message
-- with an incremented count.  When two different messages are merged,
-- the first message is logged and the second is returned as the result.
mergeEntries :: [Entry] -> [Entry] -> Writer [Entry] [Entry]
mergeEntries []   x    = return x
mergeEntries x    []   = return x
mergeEntries [e1] [e2] = let (Log n msg)   = e1
                             (Log n' msg') = e2
                         in if msg == msg' then
                              return [(Log (n+n') msg)]
                            else
                              do tell [e1]
                                 return [e2]

-- This is a complex-looking function but it is actually pretty simple.
-- It maps a function over a list of values to get a list of Writers,
-- then runs each writer and combines the results.  The result of the function
-- is a writer whose value is a list of all the values from the writers and whose
-- log output is the result of folding the merge operator into the individual
-- log entries (using 'initial' as the initial log value).
groupSame :: (Monoid a) => a -> (a -> a -> Writer a a) -> [b] -> (b -> Writer a c) -> Writer a [c]
groupSame initial merge []     _  = do tell initial
                                       return []
groupSame initial merge (x:xs) fn = do (result,output) <- return (runWriter (fn x))
                                       new             <- merge initial output
                                       rest            <- groupSame new merge xs fn
                                       return (result:rest)

-- this filters a list of packets, producing a filtered packet list and a log of
-- the activity in which consecutive messages are merged
filterAll :: [Rule] -> [Packet] -> Writer [Entry] [Packet]
filterAll rules packets = do tell [Log 1 "STARTING PACKET FILTER"]
                             out <- groupSame [] mergeEntries packets (filterOne rules)
                             tell [Log 1 "STOPPING PACKET FILTER"]
                             return (catMaybes out)
\end{minted}
\clearpage

\subsection{The Continuation monad}
\begin{itemize}[leftmargin=*,topsep=0pt,itemsep=0pt]
\item \textbf{计算的类型:}可被中断和恢复的计算
\item \textbf{Binding策略:}将函数绑到monadic的值上,产生一个将函数用作monadic计算的延续的新的延续
\item \textbf{适用于:}复杂的控制结构、错误处理、协程(co-routine)创建
\item \textbf{zero 与 plus:}无
\item \textbf{样例类型:}\texttt{Cont r a}
\end{itemize}

\subsubsection{动机}
\noindent{}\textbf{注意!}在使用continuation monad前,请确保对延续传递风格(continuation-passing-style,CPS)有坚实的
理解,并且确保延续代表了你所面对的特定问题的最佳解决方案。滥用continuation monad会导致代码无法被理解和维护。

\indent{}延续代表了一个计算的“未来”,它是从中间结果到最终结果的函数。在延续传递风格中,计算从嵌套延续的序列中建立,以
一个最后产生最终结果的延续(往往是\texttt{id})终止。因为延续是表示计算的未来的函数,对延续函数的操纵,可以实现
对计算的未来的复杂操纵。比如,在计算中间打断它、舍弃一部分计算、重启计算和交错执行计算。Continuation monad将CPS适配
到了monad的结构中。

\subsubsection{定义}
\begin{minted}[mathescape=true,
               %linenos,
               numbersep=5pt,
               autogobble,
               frame=lines,
               fontsize=\footnotesize,
               bgcolor=bg,
               framesep=2mm]{Haskell}
-- r is the final result type of the whole computation
newtype Cont r a  =  Cont { runCont :: ((a -> r) -> r) }

instance Monad (Cont r) where
    return a        =  Cont $ \k -> k a                       -- i.e. return a = \k -> k a
    (Cont c) >>= f  =  Cont $ \k -> c (\a -> runCont (f a) k) -- i.e. c >>= f = \k -> c (\a -> f a k)
\end{minted}
\noindent{}Continuation monad将计算以延续传递风格表示。\texttt{Cont r a}是一个CPS计算,它在一个最终返回类型
为\texttt{r}的CPS计算中产生类型为\texttt{a}的中间结果。

\indent{}\texttt{return}函数简单地创建了一个将值传递下去的延续。\texttt{>>=}操作将bound function加入延续链中。
\begin{minted}[mathescape=true,
               %linenos,
               numbersep=5pt,
               autogobble,
               frame=lines,
               fontsize=\footnotesize,
               bgcolor=bg,
               framesep=2mm]{Haskell}
class (Monad m) => MonadCont m where
    callCC :: ((a -> m b) -> m a) -> m a

instance MonadCont (Cont r) where
    callCC f = Cont $ \k -> runCont (f (\a -> Cont $ \_ -> k a)) k
\end{minted}
\noindent{}\texttt{MonadCont}类提供了\texttt{callCC}函数,它提供了一种逃逸延续机制。逃逸延续可使你抛弃当前
的计算,并立即返回一个值。这可以达到与\texttt{Error} monad中\texttt{throwError}和\texttt{catchError}相同的效果。

\indent{}\texttt{callCC}将当前延续作为参数来调一个函数。使用\texttt{callCC}的标准惯用法是提供一个lambda表达式
为这个延续命名,然后在它(即lambda)的scope的某个位置调用该命名延续来从计算中逃逸。

\indent{}除了\texttt{callCC}提供的逃逸机制,continuation monad还可被用来实现其它非常强力的操纵机制。不过,它们
都有非常特定的用途,而且滥用它们很容易就会使代码难以被理解,因而这里不会提及。

\subsubsection{例子}\label{sssec:cont_example}
这里我们给出一个例子,展示逃逸机制是如何工作的。这里给出的示例函数使用逃逸延续在一个整数上进行复杂的变换。
\begin{minted}[mathescape=true,
               linenos,
               numbersep=5pt,
               autogobble,
               frame=lines,
               fontsize=\footnotesize,
               bgcolor=bg,
               framesep=2mm]{Haskell}
{- We use the continuation monad to perform "escapes" from code blocks.
   This function implements a complicated control structure to process numbers:

   Input (n)       Output                       List Shown
   =========       ======                       =========
   0-9             n                            none
   10-199          number of digits in (n/2)    digits of (n/2)
   200-19999       n                            digits of (n/2)
   20000-1999999   (n/2) backwards              none
   >= 2000000      sum of digits of (n/2)       digits of (n/2)
-}

fun :: Int -> String
fun n = (`runCont` id) $ do
        str <- callCC $ \exit1 -> do                        -- define "exit1"
          when (n < 10) (exit1 (show n))
          let ns = map digitToInt (show (n `div` 2))
          n' <- callCC $ \exit2 -> do                       -- define "exit2"
            when ((length ns) < 3) (exit2 (length ns))
            when ((length ns) < 5) (exit2 n)
            when ((length ns) < 7) $ do let ns' = map intToDigit (reverse ns)
                                        exit1 (dropWhile (=='0') ns')  --escape 2 levels
            return $ sum ns
          return $ "(ns = " ++ (show ns) ++ ") " ++ (show n')
        return $ "Answer: " ++ str
\end{minted}

\clearpage

\section{现实世界中的Monad}
\indent{}我们在第一部分介绍了monad的概念,并在第二部分给出了对于那些既常见又有用的monads的一些理解。
然而,这些内容对于将monads应用于实践中仍然是不够的,因为在现实世界中你会经常希望能够做一些将几种monads的概念
结合在一起的计算。比如说,带状态的非确定性计算,或者是利用了延续而又做I/O的计算。当一个计算是另一计算的严格
子集时,分别进行monad计算是可能的,除非子计算要在单向monad中进行。

\indent{}通常,计算不会是孤立进行的。在这种情况下,我们需要的是将两种monads的特性结合进一个单独计算的那种monad。
如果每次需要一个新的计算时我们都去写带有所需特性的新的monad,将会是非常低效和糟糕的。与此相反,我们会
希望能够建立起一种将标准monads结合在一起来产生所需的混合型monad的方式。能够让我们做这件事的技术被称为monad变换器。

\indent{}Monad 变换器是本部分的主题,我们会通过重温前面的例子来看到monad变换器是如何被
使用来为这些例子增加更多现实功能的。

\subsection{用笨办法组合Monads}
\indent{}在研究monad变换器的使用之前,先来看一下如何在不使用monad变换器的情形下将monads组合起来。这是非常有用的,
它可以让我们更为深刻地了解在对monads进行组合时会出现的问题,并且提供了一个baseline以显示monad变换器的优势。
我们将使用~\ref{sssec:cont_example}~中的例子来展示这些问题。

\vspace{-0.5em}
\subsubsection{嵌套monads}\label{sssec:nested_monads}
\indent{}有些计算的结构足够简单,monadic的计算可以嵌套,避免了将monad组合在一起的需求。在Haskell中,所有发生在
IO monad中的计算都是位于顶层的,因而到目前为止我们看到的所有例子实际上都是使用了嵌套monadic计算的技术。计算在
一开始就执行所有的输入(通常是通过从命令行读参数),然后将值传递给monadic计算来产生结果,并最终在结束时执行输出。
这种结构避免了monad组合的问题,但会显得这些例子很刻意。

\indent{}前述~\ref{sssec:cont_example}~中的例子遵循了嵌套的模式:在IO monad中,从命令行读一个数,然后将这个数传给
continuation monad中的计算并产生一个字符串,接着再把这个字符串在IO monad中写回。IO monad中的计算并没有被限制为从
命令行读或写字符串,它们可以是任意复杂的。同样地,内部的计算也可以是任意复杂的。只要内部的计算不依赖于外部monad的
功能,它就可以在外部monad中安全地被嵌套。如下方的~\ref{sssec:cont_example}~例子的变体所示,它从stdin中读
而非使用命令行参数:
\vspace{-0.5em}
\begin{minted}[mathescape=true,
               linenos,
               numbersep=5pt,
               autogobble,
               frame=lines,
               fontsize=\footnotesize,
               bgcolor=bg,
               framesep=2mm]{Haskell}
fun :: IO String
fun = do n <- (readLn::IO Int)         -- this is an IO monad block
         return $ (`runCont` id) $ do  -- this is a Cont monad block
           str <- callCC $ \exit1 -> do
             when (n < 10) (exit1 (show n))
             let ns = map digitToInt (show (n `div` 2))
             n' <- callCC $ \exit2 -> do
               when ((length ns) < 3) (exit2 (length ns))
               when ((length ns) < 5) (exit2 n)
               when ((length ns) < 7) $ do let ns' = map intToDigit (reverse ns)
                                           exit1 (dropWhile (=='0') ns')
               return $ sum ns
             return $ "(ns = " ++ (show ns) ++ ") " ++ (show n')
           return $ "Answer: " ++ str
\end{minted}

\vspace{-0.8em}
\subsubsection{合并monads}\label{sssec:comb_monads_hard}
\indent{}如果是带有更加复杂的结构的计算呢?如果嵌套模式不能够被使用,我们就需要一种将两个或多个monads的属性
结合在一个单独计算中的方法。这是通过在这样一个monad中做计算做到的,这个monad中的值本身也是另一个monad中的
monadic值。举例来说,如果我们需要在continuation monad中的计算中做I/O的话,我们可能会想要在类型为
\texttt{Cont (IO String) a}的continuation monad中做计算。我们可以使用一个类型为\texttt{State (Either Err a) a}的
monad以在一个单独的计算中组合state monad与error monad的特性。

\indent{}考虑修改一下我们的例子:我们在开始时仍旧做相同的I/O,但在continuation monad中的计算的中途,我们可能
会需要额外的输入。在这种情况下,当输入值在一定范围内时,我们将允许用户指定部分输出值。由于I/O依赖于
continuation monad中的部分计算,而该计算又依赖于I/O的结果,所以我们不能使用嵌套monad模式。

\indent{}与之不同,我们会这样做:让continuation monad中的计算使用IO monad的值。过去是\texttt{Int}和\texttt{String}类型的
值,现在则成为了类型为\texttt{IO Int}与\texttt{IO String}类型的值。我们不可以从IO monad中将值抽出(因为是单向monad),因而
我们可能需要稍稍在continuation monad中嵌套IO monad的do block以操纵值。这里我们使用了一个辅助函数\texttt{toIO}。当我们
创建位于在continuation monad中嵌套的IO monads中的值时,它会使过程显得更加清晰一些。
\begin{minted}[mathescape=true,
               linenos,
               numbersep=5pt,
               autogobble,
               frame=lines,
               fontsize=\footnotesize,
               bgcolor=bg,
               framesep=2mm]{Haskell}
toIO :: a -> IO a
toIO x = return x

fun :: IO String
fun = do n <- (readLn::IO Int)         -- this is an IO monad block
         convert n

convert :: Int -> IO String
convert n = (`runCont` id) $ do        -- this is a Cont monad block
              str <- callCC $ \exit1 -> do    -- str has type IO String
                when (n < 10) (exit1 $ toIO (show n))
                let ns = map digitToInt (show (n `div` 2))
                n' <- callCC $ \exit2 -> do   -- n' has type IO Int
                  when ((length ns) < 3) (exit2 (toIO (length ns)))
                  when ((length ns) < 5) (exit2 $ do putStrLn "Enter a number:"
                                                     x <- (readLn::IO Int)
                                                     return x)
                  when ((length ns) < 7) $ do let ns' = map intToDigit (reverse ns)
                                              exit1 $ toIO (dropWhile (=='0') ns')
                  return (toIO (sum ns))
                return $ do num <- n'  -- this is an IO monad block
                            return $ "(ns = " ++ (show ns) ++ ") " ++ (show num)
              return $ do s <- str -- this is an IO monad block
                          return $ "Answer: " ++ s
\end{minted}

\subsection{Monad变换器}
\indent{}在~\ref{sssec:comb_monads_hard}~中,我们以一种方式将monads结合在一起。同时我们也可以看到,即使是那样一个
trivial的例子就已经十分混乱与丑陋了,那样的做法确实是很糟糕的,尽管它能起作用。
将~\ref{sssec:nested_monads}~与~\ref{sssec:comb_monads_hard}~中的两块代码并排比较一下,就能看到手动对monads进行组合的
策略对代码的污染性。

\indent{}Monad变换器是标准monads的特殊变体,它们为monads的组合提供了便利。它们的类型构造器是在monad类型构造器上参数化的,
并且产生了合并的monadic类型。

\vspace{-0.5em}
\subsubsection{变换器类型构造器(Transformer type constructors)}
\indent{}就Haskell对monad的支持而言,类型构造器是其基础构件。回想一下,\texttt{Reader r a}是这样一种值的类型:
带有类型为\texttt{r}的环境的reader monad内的类型为\texttt{a}的值。类型构造器\texttt{Reader r}是\texttt{Monad}类的
实例,并且函数\texttt{runReader::(r->a)}执行reader monad中的计算并返回类型为\texttt{a}的值。

\indent{}所谓的\texttt{ReaderT}是reader monad的变换器版本,它含有一个作为额外参数加入的monad类型构造器。
\texttt{ReaderT r m a}是这样一种组合monad的值的类型:reader是基-monad,且\texttt{m}是其内部的monad。
\texttt{ReaderT r m}是\texttt{Monad}类的实例,且函数\texttt{runReaderT::(r -> m a)}执行组合monad中的计算,
并返回类型为\texttt{m a}的结果。

\indent{}我们可以使用monads的变换器版本来轻松地创建组合monads。比如,\texttt{ReaderT r IO}组合了reader monad和
IO monad。我们也可以通过将monads的变换器版本应用到Identity monad上来生成与之对应的monad的非变换器版本。比如,
\texttt{ReaderT r Identity}与\texttt{Reader r}是相同的。

\vspace{-0.5em}
\subsubsection{Lifting}
\indent{}当使用通过monad变换器创建的组合monad时,我们不需要显式地对内部的monad类型进行管理,这使得我们的代码
更加清晰简单。我们不再通过在计算中添加额外的do block来操纵内部monad类型中的值,而是可以使用lifting操作将
函数从内部monad带至组合monad。

\indent{}回想\texttt{liftM}函数族,它们用于将非monadic函数lift进monad中。每个monad变换器都提供了\texttt{lift}函数,
它用于将一个monadic计算lift进组合monad中。许多变换器还提供了\texttt{liftIO}函数,这是为IO monad提供的\texttt{lift}的
优化版本。现在,让我们再来改写之前的例子:
\begin{minted}[mathescape=true,
               linenos,
               numbersep=5pt,
               autogobble,
               frame=lines,
               fontsize=\footnotesize,
               bgcolor=bg,
               framesep=2mm]{Haskell}
fun :: IO String
fun = (`runContT` return) $ do
        n   <- liftIO (readLn::IO Int)
        str <- callCC $ \exit1 -> do     -- define "exit1"
          when (n < 10) (exit1 (show n))
          let ns = map digitToInt (show (n `div` 2))
          n' <- callCC $ \exit2 -> do    -- define "exit2"
            when ((length ns) < 3) (exit2 (length ns))
            when ((length ns) < 5) $ do liftIO $ putStrLn "Enter a number:"
                                        x <- liftIO (readLn::IO Int)
                                        exit2 x
            when ((length ns) < 7) $ do let ns' = map intToDigit (reverse ns)
                                        exit1 (dropWhile (=='0') ns')  --escape 2 levels
            return $ sum ns
          return $ "(ns = " ++ (show ns) ++ ") " ++ (show n')
        return $ "Answer: " ++ str
\end{minted}

\subsection{标准Monad变换器}
\indent{}Haskell的基础库以类的形式为monad变换器提供了支持,这些类表示了monad变换器与标准monad的特别变换器版本。

\subsubsection{MonadTrans 与 MonadIO}
\indent{}定义在\texttt{Control.Monad.Trans}中的\texttt{MonadTrans}类只提供了一个函数:\texttt{lift}。就像我们
在上面说过的,它将一个内部monad中的monadic计算lift到组合monad中。
\begin{minted}[mathescape=true,
               %linenos,
               numbersep=5pt,
               autogobble,
               %frame=lines,
               fontsize=\footnotesize,
               bgcolor=bg,
               framesep=2mm]{Haskell}
-- | The class of monad transformers.  Instances should satisfy the
-- following laws, which state that 'lift' is a monad transformation:
--     * @'lift' . 'return' = 'return'@
--     * @'lift' (m >>= f) = 'lift' m >>= ('lift' . f)@
class MonadTrans t where
    lift :: (Monad m) => m a -> t m a
\end{minted}

\indent{}\texttt{MonadIO}类中的函数\texttt{liftIO}为IO操作的lifting提供了一个\texttt{lift}的优化版本。
\begin{minted}[mathescape=true,
               %linenos,
               numbersep=5pt,
               autogobble,
               %frame=lines,
               fontsize=\footnotesize,
               bgcolor=bg,
               framesep=2mm]{Haskell}
-- | Monads in which 'IO' computations may be embedded.
class (Monad m) => MonadIO m where
    liftIO :: IO a -> m a

-- | @since 4.9.0.0
instance MonadIO IO where
    liftIO = id
\end{minted}

\subsubsection{标准Monads的变换器版本}
\indent{}Monad模板库中的标准monads都有与非变换器版本对应的变换器版本。不过,并不是所有的monad变换器都做
相同的变换。我们已经看到,\texttt{contT}变换器将形式为\texttt{(a->r) -> r}的延续变为形式为\texttt{(a->m r) -> m r}
的延续。然而,\texttt{StateT}变换器与此不同,它将形式为\texttt{s -> (a,s)}的状态变换器函数变为形式为
\texttt{s -> m (a,s)}的状态变换器函数。通常来说,并没有用于创建变换器版本的通用法则,它们的形式取决于其实际意义。

\indent{}这里给出一份关于标准monad变换器的表格:
\begin{table}[h]
\begin{center}
\resizebox{0.6\textwidth}{!}{%
\begin{tabular}{cccc}
\hline
\textbf{标准Monad} & \textbf{变换器版本} & \textbf{原类型} & \textbf{合并类型} \\ \hline\hline
Error & ErrorT & \texttt{Either e a} & \texttt{m (Either e a)} \\
State & StateT & \texttt{s -> (a,s)} & \texttt{s -> m (a,s)} \\
Reader & ReaderT & \texttt{r -> a} & \texttt{r -> m a} \\
Writer & WriterT & \texttt{(a,w)} & \texttt{m (a,w)} \\
Cont & ContT & \texttt{(a -> r) -> r} & \texttt{(a -> m r) -> m r}\\ \hline
\end{tabular}}
\end{center}
\end{table}

\indent{}注意,在做monad合并时,顺序是很重要的。\texttt{StateT s (Error e)}与\texttt{ErrorT e (State s)}是
不同的。前者会产生\texttt{s -> Error e (a,s)}的组合类型,其中计算要么返回一个新的状态,要么产生一个错误。
而后者会产生\texttt{s -> (Error e a,s)}的组合类型,其中计算总是返回一个新的状态,而值可以是一个错误或一个正常的值。

\subsection{剖析Monad变换器}
\indent{}在这一节中,我们将详细地看一下标准库中最有趣的变换器之一\texttt{StateT}的实现。研究这个变换器将会使你建立起
对你所能在代码中用到的monad变换器机制的深入理解。

\vspace{-0.5em}
\subsubsection{组合的monad定义}
\noindent{}就像之前state monad在如下定义上建立一样:
\begin{minted}[mathescape=true,
               %linenos,
               numbersep=5pt,
               autogobble,
               %frame=lines,
               fontsize=\footnotesize,
               bgcolor=bg,
               framesep=2mm]{Haskell}
newtype State s a = State { runState :: (s -> (a,s)) }
\end{minted}
\noindent{}变换器 StateT 在下面的定义上建立:
\begin{minted}[mathescape=true,
               %linenos,
               numbersep=5pt,
               autogobble,
               %frame=lines,
               fontsize=\footnotesize,
               bgcolor=bg,
               framesep=2mm]{Haskell}
newtype StateT s m a = StateT { runStateT :: (s -> m (a,s)) }
\end{minted}
\noindent{}\texttt{State s}既是\texttt{Monad}类的实例,也是\texttt{MonadState s}类的实例,因而
\texttt{StateT s m}应该也一样是它们的实例。并且,如果\texttt{m}是\texttt{MonadPlus}的实例,那么
\texttt{StateT s m}也应该是它的实例。
\begin{minted}[mathescape=true,
               %linenos,
               numbersep=5pt,
               autogobble,
               %frame=lines,
               fontsize=\footnotesize,
               bgcolor=bg,
               framesep=2mm]{Haskell}
newtype StateT s m a = StateT { runStateT :: (s -> m (a,s)) }

instance (Monad m) => Monad (StateT s m) where
    return a          = StateT $ \s -> return (a,s)
    (StateT x) >>= f  = StateT $ \s -> do (v,s')      <- x s
                                          (StateT x') <- return $ f v
                                          x' s'

instance (Monad m) => MonadState s (StateT s m) where
    get   = StateT $ \s -> return (s,s)
    put s = StateT $ \_ -> return ((),s)
\end{minted}
\noindent{}将此与\texttt{State s}的定义对比,\texttt{return}使用了其内部monad的\texttt{return},并且
使用了一个do block以在其内部monad中执行计算。别忘了\texttt{StateT s m}也是\texttt{MonadPlus}的实例:
\begin{minted}[mathescape=true,
               %linenos,
               numbersep=5pt,
               autogobble,
               %frame=lines,
               fontsize=\footnotesize,
               bgcolor=bg,
               framesep=2mm]{Haskell}
instance (MonadPlus m) => MonadPlus (StateT s m) where
    mzero = StateT $ \s -> mzero
    (StateT x1) `mplus` (StateT x2) = StateT $ \s -> (x1 s) `mplus` (x2 s)
\end{minted}

\vspace{-0.5em}
\subsubsection{定义lifting函数}
\noindent{}最后我们还要提供\texttt{lift}函数,以使\texttt{StateT s}成为\texttt{MonadTrans}的实例:
\begin{minted}[mathescape=true,
               %linenos,
               numbersep=5pt,
               autogobble,
               %frame=lines,
               fontsize=\footnotesize,
               bgcolor=bg,
               framesep=2mm]{Haskell}
instance MonadTrans (StateT s) where
    lift c = StateT $ \s -> c >>= (\x -> return (x,s))
\end{minted}
\indent{}\texttt{lift}函数创建了一个\texttt{StateT}状态变换器函数,这个函数将内部monad中的计算绑定到
一个将结果与输入状态打包的函数。结果是,一个返回list(即list monad中的计算)的函数可以
被lift到\texttt{StateT s []}中,在那里它变成了一个返回\texttt{StateT (s -> [(a,s)])}的函数。也就是说,
被提升(lifted)的计算会从其输入状态产生多个\texttt{(value,state)}对。其效果是去``fork'' StateT中的计算,
并产生计算的一个(对于被提升的函数返回的list中的每个值的)不同分支。当然,将\texttt{StateT}应用到不同的
monad上会为\texttt{lift}函数赋予不同的语义。


\end{document}