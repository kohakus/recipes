\documentclass[12pt]{article}
\usepackage{array}
\usepackage{amsmath}
\usepackage{amssymb}
\usepackage{algorithm}
\usepackage{algorithmic}
\usepackage{caption}
\usepackage{fontspec}
\usepackage{graphicx}
\usepackage{indentfirst}
\usepackage{minted}
\usepackage{mathtools}
\usepackage{pifont}
\usepackage{setspace}
\usepackage{subfigure}
\usepackage{tikz}
\usepackage{url}
\usepackage{xcolor}
\usepackage{xeCJK}

\usepackage[colorlinks=true]{hyperref}
\usepackage[margin=0.55in]{geometry}

% background color for minted
\definecolor{bg}{rgb}{0.95,0.95,0.95}

% CJK font
\setCJKmainfont{Source Han Serif CN}

% indent value
\setlength{\parindent}{2em}

% line spacing
\linespread{1.2}
\renewcommand{\thesection}{\Roman{section}}
\renewcommand{\thesubsection}{\thesection-\arabic{subsection}}

\title{The Essence of Functional Programming}
\author{Yiteng Zhang}

\begin{document}
\maketitle

\noindent{}\textbf{注:}本文是对Philip Wadler的经典文章\textit{The Essence of Functional Programming}(函数
式编程的本质)的简译和总结,包含其文章主要的内容,并略去了一部分相关讨论。具体的讨论细节可以结合近年来
相关主题的发展(包括Philip Wadler文集)进一步了解。
~\\

\noindent{}\textbf{摘要:}这篇文章探索了利用monads(单子)来构建函数式程序,而不涉及有关monads和范畴论
的预备知识。Monads 增强了程序修改的 便利性,它们可以模拟不纯(impure)特性的效用,比如
exception(异常),state(状态)与continuation(延续),并可提
供这些特性难以轻易达成的效用。一个程序的类型(types)反映了具体发生了什么效用。
文章的第一部分是一个有关使用monads的扩展
例子,一个简单的解释器被修改,以支持各种额外的特性:错误信息、状态、输出与
非确定性选择(non-deterministic choice)。第二部分描述了
monads与延续传递风格(continuation-passing style)之间的关系。第三部分勾勒了monads是如何被用于一个由Haskell
写成的Haskell编译器中的。

\section{Introduction}
\noindent{}\textbf{纯(pure)还是不纯(impure),如何选择?}

\indent{}纯函数式语言,比如Haskell,提供了惰性求值(lazy evaluation)的能力与等式推理(equational resoning)的
简单性。不纯的函数式语言,比如Standard ML或Scheme,提供了一些诱人的特性,比如状态,异常处理和延续。一个影响
选择的因素是对应的程序能够被修改的简易性。纯函数式语言通过使每个操作所依赖的数据显性化来简化更改。但有时,一
个看似很小的变更就需要对纯函数式程序进行大规模重构,而明智地使用不纯特性时,只需修改少数几行即可。

\indent{}假设我们用纯函数式语言写了一个解释器。为了给它加上异常处理,我们需要修改结果类型以包含错误值,并在
每个递归调用中检查和正确处理错误。当使用不纯语言的异常处理时,就不需要这种重构。为了给它加上调用计数,我们
需要修改结果类型以包含计数值,并修改每次的递归调用来正确地传递这个计数。若使用不纯语言,就可以直接用一个全局变量
作计数用。为了给它加上输出指令,我们需要修改结果类型以包含输出列表,并对每次递归调用进行修改来正确地传递这个列表。
如果使用不纯的语言,输出即为一种副作用,就不需要做那种重构。
这些例子更加凸显了monad的作用。这篇文章展示了如何使用monads来构建解释器,从而使得上面提到
的那些改变都易于进行。在每种情形下,所有需要做的事只是重新定义monad并在局部处做寥寥修改。这种编程风格
可以让我们在不使用相应的不纯特性时,重获不纯语言的各种特性所能提供的灵活性。
这种技术不仅仅可以应用在解释器上,其实它可以在函数式程序中有很广泛的应用。位于Glasgow的GRASP
团队为名为Haskell的函数式语言构建了一个编译器,这个编译器本身也是用Haskell写成的,并使用了monads。

\indent{}使用monads进行编程很难不让人联想到延续传递风格(continuation-passing style,CPS),这篇文章也探讨了它们二者
之间的关系。某种情形下它们是等价的:CPS作为monad的一种特例出现,而任何monad也可能通过改变应答类型(answer type)被
嵌入CPS中。但是,monadic的方法提供了额外的insight,并能够提供更精细程度上的控制。

\indent{}关于标题,需要注意的是,使用“essence”一词的目的是在技术层面上强调这篇文章中描述的技术是很有用的,但并不是
说他们是必不可少的。
%余下章节是这样组织的:第二章通过考虑一个解释器的不同变体来描述monads在构建程序时的使用;
%第三章探索了monads和CPS之间的关系;第四章勾勒了这种思想是如何被用于Haskell编译器中的。
\vspace{-0.6em}
\section{解释Monads}\label{sec:interp_monad}
这一章通过展示一个用作lambda演算的解释器的几个变体来佐证monads增强了模块化这一论点。
该解释器如下所示,这里采用了Haskell进行描述。
\begin{minted}[mathescape=true,
               %linenos,
               numbersep=5pt,
               autogobble,
               frame=lines,
               fontsize=\scriptsize,
               bgcolor=bg,
               framesep=2mm]{Haskell}
type  Name             =  String

data  Term             =  Var Name
                       |  Con Int
                       |  Add Term Term
                       |  Lam Name Term
                       |  App Term Term

data  Value            =  Wrong
                       |  Num Int
                       |  Fun (Value -> M Value)

type  Environment      =  [(Name, Value)]

showval                :: Value -> String
showval Wrong          =  "<wrong>"
showval (Num i)        =  showint i
showval (Fun f)        =  "<function>"

interp                 :: Term -> Environment -> M Value
interp (Var x) e       =  lookup x e
interp (Con i) e       =  unitM (Num i)
interp (Add u v) e     =  interp u e `bindM` (\a ->
                          interp v e `bindM` (\b ->
                          add a b))
interp (Lam x v) e     =  unitM (Fun (\a -> interp v ((x,a):e)))
interp (App t u) e     =  interp t e `bindM` (\f ->
                          interp u e `bindM` (\a ->
                          apply f a))

lookup                 :: Name -> Environment -> M Value
lookup x []            =  unitM Wrong
lookup x ((y,b):e)     =  if x==y then unitM b else lookup x e

add                    :: Value -> Value -> M Value
add (Num i) (Num j)    =  unitM (Num (i+j))
add a b                =  unitM Wrong

apply                  :: Value -> Value -> M Value
apply (Fun k) a        =  k a
apply f a              =  unitM Wrong

test                   :: Term -> String
test t                 =  showM (interp t [])
\end{minted}
\noindent{}其中,类型构造器(type constructor)\texttt{M}和函数\texttt{unitM},\texttt{bindM}与\texttt{showM}都
与monads有关,后面会有相应的解释。解释器处理值(values)和terms。一个值要么是\texttt{Wrong},要么是一个数字,或
是一个函数。\texttt{Wrong}值代表一个错误,比如未绑定的变量,一次非数字的相加,或者一次非函数的应用。一个
term可以是一个变量,或一个常量,或一个和,或一个lambda表达式,或是一次应用。

\indent{}比如我们有一个term $((\lambda x.\,x + x) (10 + 11))$,记为\texttt{term0},用我们的解释器来描述即为:
\begin{minted}[mathescape=true,
               %linenos,
               numbersep=5pt,
               autogobble,
               frame=lines,
               fontsize=\footnotesize,
               bgcolor=bg,
               framesep=2mm]{Haskell}
term0  =  (App (Lam "x" (Add (Var "x") (Var "x")))
               (Add (Con 10) (Con 11)))
\end{minted}
\noindent{}对于标准的解释器,对\texttt{test term0}进行求值会得到字符串\texttt{"42"}。

\indent{}为方便说明起见,这个解释器比较小。它可以很容易被扩充以应对额外的值,比如布尔值,pairs和lists;还有
额外的term形式,比如说条件(conditional)与不动点(fixpoint)。

\subsection{什么是monad?}\label{sec:what_is_monad}
\indent{}就我们的目的而言,一个monad是一个三元组(triple):\texttt{(M, unitM, bindM)},其包含
一个类型构造器(type constructor)\texttt{M},和一对儿多态(polymorphic)函数。
\begin{minted}[mathescape=true,
               %linenos,
               numbersep=5pt,
               autogobble,
               frame=lines,
               fontsize=\footnotesize,
               bgcolor=bg,
               framesep=2mm]{Haskell}
unitM  ::  a -> M a
bindM  ::  M a -> (a -> M b) -> M b
\end{minted}
\noindent{}这些函数必须满足三条定律,见 \ref{sec:monad_law} 小节。

\indent{}将一个程序转换为monadic形式的基本思路是:一个类型为\texttt{a -> b}的函数被转到类型为\texttt{a -> M b}的
函数。因而,在\texttt{Value}的定义中,函数的类型是\texttt{Value -> M Value} 而非\texttt{Value -> Value},并且
\texttt{interp}的类型为\texttt{Term -> Environment -> M Value} 而非\texttt{Term -> Environment -> Value}。
与其相似的,在辅助函数\texttt{lookup},\texttt{add}和\texttt{apply}中也都是如此。

\indent{}恒等函数的类型是\texttt{a -> a},对应的monadic的函数是\texttt{unitM :: a -> M a}。它取一个值,将其
变成相应的monad形式的值。当我们考虑常数的情况时:
\begin{minted}[mathescape=true,
               %linenos,
               numbersep=5pt,
               autogobble,
               frame=lines,
               fontsize=\footnotesize,
               bgcolor=bg,
               framesep=2mm]{Haskell}
interp (Con i) e  =  unitM (Num i)
\end{minted}
\noindent{}表达式\texttt{(Num i)}的类型是\texttt{Value},因而将\texttt{unitM}作用于其上,
会得到相应的\texttt{M Value}。

\indent{}两个函数\texttt{k :: a -> b}和\texttt{h :: b -> c}通过写成如下形式被组合(composed)起来:
\begin{minted}[mathescape=true,
               %linenos,
               numbersep=5pt,
               autogobble,
               frame=lines,
               fontsize=\footnotesize,
               bgcolor=bg,
               framesep=2mm]{Haskell}
\a -> let b = k a in h b
\end{minted}
\noindent{}这个函数的类型是\texttt{a -> c}。(这里 \texttt{\textbackslash{}name -> expr}是lambda表达式)
与此相似的,两个在monadic形式下的函数\texttt{k :: a -> M b}和 \texttt{h :: b -> M c}的组合可以写成:
\begin{minted}[mathescape=true,
               %linenos,
               numbersep=5pt,
               autogobble,
               frame=lines,
               fontsize=\footnotesize,
               bgcolor=bg,
               framesep=2mm]{Haskell}
(\a -> k a `bindM` (\b -> h b))
\end{minted}
\noindent{}其具有类型\texttt{a -> M c}。因而,\texttt{bindM}在这里的角色和\texttt{let}表达式有些像。
上面提到的三条monad定律只是保证这个组合形式是可结合的,并以\texttt{unitM}为左单位元与右单位元(identity)。

\indent{}考虑如下的求和情形:
\begin{minted}[mathescape=true,
               %linenos,
               numbersep=5pt,
               autogobble,
               frame=lines,
               fontsize=\footnotesize,
               bgcolor=bg,
               framesep=2mm]{Haskell}
interp (Add u v) e  =  interp u e `bindM` (\a ->
                       interp v e `bindM` (\b ->
                       add a b))
\end{minted}
\indent{}上面的片段可以这样读:对\texttt{u}求值;将\texttt{a}绑定到其结果上;对\texttt{v}求值,将\texttt{b}绑定
到其结果上;将\texttt{a}加到\texttt{b}上。这样类型也就出来了:对\texttt{interp}与\texttt{add}的调用给出的结果
具有类型\texttt{M Value},变量\texttt{a}和\texttt{b}具有类型\texttt{Value}。

\indent{}对函数应用的处理是类似的。特别是,函数和它的参数会被求值,所以这个解释器使用的是call-by-value策略。一个
使用call-by-name策略的解释器会在第二章被讨论到。

\indent{}就像类型\texttt{Value}表示一个值那样,类型\texttt{M Value}可以被看作是表示一个计算。\texttt{unitM}的
目的是将一个值压到一个计算里。\texttt{bindM}的目的是对一个计算求值,并给出一个值。

\indent{}不严格地来讲,\texttt{unitM}将我们带入monad,\texttt{bindM}让我们绕着monad。那我们怎么从monad里出来呢?
通常来说,这样的操作需要一个更为特别的设计。就我们的目的而言,要提供下面这个:
\begin{minted}[mathescape=true,
               %linenos,
               numbersep=5pt,
               autogobble,
               frame=lines,
               fontsize=\footnotesize,
               bgcolor=bg,
               framesep=2mm]{Haskell}
showM :: M Value -> String
\end{minted}
\noindent{}它在\texttt{test}中被使用。

\indent{}通过改变\texttt{M}、\texttt{unitM}、\texttt{bindM}与\texttt{showM}的定义,并做一些其它的小修小改,
这个解释器就可以表现出各种各样的行为,这一点正是下面的各种变体将要展示的。

\vspace{-0.3em}
\subsection{变体零:标准的解释器}
\noindent{}作为起始,先来定义一个平凡的monad。
\begin{minted}[mathescape=true,
               %linenos,
               numbersep=5pt,
               autogobble,
               frame=lines,
               fontsize=\footnotesize,
               bgcolor=bg,
               framesep=2mm]{Haskell}
type  I a      =  a

unitI a        =  a
a `bindI` k    =  k a
showI a        =  showval a
\end{minted}
\noindent{}这被称之为\textsl{identity} monad:\texttt{I}是类型上的恒等函数(identity function),\texttt{unitI}是
恒等函数,\texttt{bindI}是后缀(函数)应用,\texttt{showI}与\texttt{showval}等同。

\indent{}将解释器中的monad \texttt{M}用monad \texttt{I}代换
掉(也就是说,用\texttt{I},\texttt{unitI},\texttt{bindI},\texttt{showI}将
\texttt{M},\texttt{unitM},\texttt{bindM},\texttt{showM}分别代换掉),这给出了用于lambda演算的标准
元循环(meta-circular)解释器:
\begin{minted}[mathescape=true,
               %linenos,
               numbersep=5pt,
               autogobble,
               frame=lines,
               fontsize=\footnotesize,
               bgcolor=bg,
               framesep=2mm]{Haskell}
interp                :: Term -> Environment -> Value
interp (Var x) e      =  lookup x e
interp (Con i) e      =  Num i
interp (Add u v) e    =  add (interp u e) (interp v e)
interp (Lam x v) e    =  Fun (\a -> interp v ((x,a):e))
interp (App t u) e    =  apply (interp t e) (interp u e)
\end{minted}
\noindent{}解释器中的其它函数也以相似的方式进行简化。

\subsection{变体一:错误消息}\label{sec:err_msg}
\noindent{}为了给解释器加上错误消息,定义下述monad。
\begin{minted}[mathescape=true,
               %linenos,
               numbersep=5pt,
               autogobble,
               frame=lines,
               fontsize=\footnotesize,
               bgcolor=bg,
               framesep=2mm]{Haskell}
data  E a                =  Success a | Error String

unitE a                  =  Success a
errorE s                 =  Error s

(Success a) `bindE` k    =  k a
(Error s)   `bindE` k    =  Error s

showE (Success a)        =  "Success: " ++ showval a
showE (Error s)          =  "Error: " ++ s
\end{minted}
\noindent{}解释器中的每个函数要么通过给出一个形式为\texttt{Success a}的值正常返回,要么通过给出形式
为\texttt{Error s}的值来指示一个错误,其中\texttt{s}是一条错误信息字符串。如果有\texttt{m :: E a}和
\texttt{k :: a -> E b},那么\texttt{m `bindE` k}的行为是严格的后缀应用:如果\texttt{m}成功,那么
\texttt{k}会被应用在成功的结果上;如果\texttt{m}失败了,那么应用也会失败。show函数会打印
结果(当成功时),或者是错误信息。

\indent{}在解释器中,将monad \texttt{M}用monad \texttt{E}代换掉,然后把每个出现的\texttt{unitE Wrong}
替换为\texttt{errorE}的合适调用。这些修改发生在\texttt{lookup},\texttt{add}和\texttt{apply}中,其它
地方不用动。
\begin{minted}[mathescape=true,
               %linenos,
               numbersep=5pt,
               autogobble,
               frame=lines,
               fontsize=\footnotesize,
               bgcolor=bg,
               framesep=2mm]{Haskell}
lookup x []    = errorE ("unbound variable: "   ++ x)
add a b        = errorE ("should be numbers: "  ++ showval a
                                         ++ "," ++ showval b)
apply f a      = errorE ("should be function: " ++ showval f)
\end{minted}
\noindent{}现在,对\texttt{test term0}求值会返回\texttt{"Success: 42"}。

\indent{}在不纯的语言中,这种修改可以通过使用一些特性,比如异常或者延续来达到指示错误的效用。而在
这里,我们的 monad \texttt{E}能够很好地对错误进行响应,例如对\texttt{test (App (Con 1) (Con 2))}求值的话,会
得到一条消息\texttt{"Error: should be function: 1"}。

\subsection{变体二:带有位置的错误消息}\label{sec:err_msg_pos}
\indent{}令\texttt{Position}为用来指示源代码位置(即行号)的一个类型。我们用一个指示位置的构造器(constructor)
来扩充\texttt{Term}数据类型(datatype)。
\begin{minted}[mathescape=true,
               %linenos,
               numbersep=5pt,
               autogobble,
               frame=lines,
               fontsize=\footnotesize,
               bgcolor=bg,
               framesep=2mm]{Haskell}
data  Term  =  ... | At Position Term
\end{minted}
\noindent{}解析器(parser)会产生合适的term。比如说,\texttt{(At p (App t (At q u)))}表示\texttt{p}是
term \texttt{(App t u)}的位置,且\texttt{q}是sub-term \texttt{u}的位置。

\indent{}基于monad \texttt{E},我们定义一个新的monad \texttt{P},它接受一个位置,并将其用于错误报告消息中。
这种带有位置信息的错误消息可以被看成是加强版的monad \texttt{E},定义如下:
\begin{minted}[mathescape=true,
               %linenos,
               numbersep=5pt,
               autogobble,
               frame=lines,
               fontsize=\footnotesize,
               bgcolor=bg,
               framesep=2mm]{Haskell}
type  P a      =  Position -> E a

unitP a        =  \p -> unitE a
errorP s       =  \p -> errorE (showpos p ++ ": " ++ s)

m `bindP` k    =  \p -> m p `bindE` (\x -> k x p)

showP m        =  showE (m pos0)
\end{minted}
\noindent{}其中,\texttt{unitP}会丢掉当前位置,而\texttt{errorP}会将当前位置加在错误消息上,\texttt{bindP}将
位置传给参数和函数,\texttt{showP}会传入一个初始位置\texttt{pos0}。除此之外,还有一个用于改变位置的函数:
\begin{minted}[mathescape=true,
               %linenos,
               numbersep=5pt,
               autogobble,
               frame=lines,
               fontsize=\footnotesize,
               bgcolor=bg,
               framesep=2mm]{Haskell}
resetP        ::  Position -> P x -> P x
resetP q m    =  \p -> m q
\end{minted}
\noindent{}它会丢掉传入的位置\texttt{p},并用给定的位置\texttt{q}替代它。

\indent{}要修改之前版本解释器,我们将解释器中的每处monad \texttt{E} 替换为monad \texttt{P},并增加一
种对\texttt{At} term进行处理的case:
\begin{minted}[mathescape=true,
               %linenos,
               numbersep=5pt,
               autogobble,
               frame=lines,
               fontsize=\footnotesize,
               bgcolor=bg,
               framesep=2mm]{Haskell}
interp (At p t) e  =  resetP p (interp t e)
\end{minted}
\noindent{}这会将位置重置。其它的地方无需改动。

\indent{}如果没有monad或其他类似的技术,针对这种效用的修改将会是非常枯燥而乏味的。在那种情形下,
解释器中每一块相关的地方都要被重写,以将当前位置作为额外的参数接受,并将其在每个递归调用时传递。
在不纯的语言中,想要达到类似效用所需的修改也不是那么简单的。一个方法是使用一个用于保存位置
的(栈结构的)状态变量。必须要小心而正确地维护状态:在进入\texttt{At}结构时将位置推入栈中,在
离开时将位置从栈中弹出。

\subsection{变体三:状态}\label{sec:variation_state}
\indent{}为了描绘对状态的操纵,解释器会被修改以记录在计算过程中发生归约(reduction)的次数。
相同的技术可以用于处理其他状态依赖(state-dependent)的构造,比如用引用值(reference values)和对堆
有副作用的操作来扩展解释型语言(interpreted language)。

\indent{}状态变换器的monad定义如下:
\begin{minted}[mathescape=true,
               %linenos,
               numbersep=5pt,
               autogobble,
               frame=lines,
               fontsize=\footnotesize,
               bgcolor=bg,
               framesep=2mm]{Haskell}
type  S a      =  State -> (a, State)

unitS a        =  \s0 -> (a, s0)
m `bindS` k    =  \s0 -> let (a, s1) = m s0
                             (b, s2) = k a s1
                         in  (b, s2)
\end{minted}

\indent{}状态转换器取一个初始状态并返回一个值,该值与新的状态一起组对。函数unit返回给定值,并将状态
不加改变地进行传播。函数bind取一个状态转换器\texttt{m :: S a}和一个函数\texttt{k :: a -> S b},它将初始状态传给
\texttt{m},这给出了一个与中间状态成组(paired)的值。函数\texttt{k}被应用到这个值上,产生一个状态转换器
\texttt{(k a :: S b)},它将中间状态进行处理并给出与最终状态成组的结果。

\indent{}为了对执行计数(execution counts)进行建模,我们取状态为整数(integer)。
\begin{minted}[mathescape=true,
               %linenos,
               numbersep=5pt,
               autogobble,
               frame=lines,
               fontsize=\footnotesize,
               bgcolor=bg,
               framesep=2mm]{Haskell}
type  State  =  Int
\end{minted}

\indent{}show函数接受初始状态(计数值为零),并打印最终状态(如上所述,在这里是计数值)。
\begin{minted}[mathescape=true,
               %linenos,
               numbersep=5pt,
               autogobble,
               frame=lines,
               fontsize=\footnotesize,
               bgcolor=bg,
               framesep=2mm]{Haskell}
showS m      =  let (a, s1) = m 0
                in  "Value: " ++ showval a ++ "; " ++
                    "Count: " ++ showval s1
\end{minted}

\indent{}当前计数值会以下述方式进行递增:
\begin{minted}[mathescape=true,
               %linenos,
               numbersep=5pt,
               autogobble,
               frame=lines,
               fontsize=\footnotesize,
               bgcolor=bg,
               framesep=2mm]{Haskell}
tickS        :: S ()
tickS        =  \s -> ((), s+1)
\end{minted}
\noindent{}其返回的值是空的元组(tuple),和值一样,其类型亦被写作\texttt{()}。\texttt{tickS}的类型说明我们并不关心
具体返回了什么值,这类似于在不纯的语言中使用一个结果类型(result type)为\texttt{()}的函数。这表明了该函数的
目的在于副作用。

\indent{}我们通过将解释器中的monad \texttt{M} 替换为monad \texttt{S}对其进行修改。具体来说,我们需要修改函数
\texttt{apply}和\texttt{add}的第一行。
\begin{minted}[mathescape=true,
               %linenos,
               numbersep=5pt,
               autogobble,
               frame=lines,
               fontsize=\footnotesize,
               bgcolor=bg,
               framesep=2mm]{Haskell}
apply (Fun k) a        =  tickS `bindS` (\() -> k a)
add (Num i) (Num j)    =  tickS `bindS` (\() -> unitS (Num (i + j)))
\end{minted}
\noindent{}对于每个应用与相加操作,都会计数一次。比如,如果我们对\texttt{test term0}求值,会得到
\texttt{"Value: 42; Count: 3"}。没有其它的地方需要被修改了。

\indent{}一个进一步的修改会将语言扩展,以允许对当前的执行计数进行访问。首先,我们给monad增加一个新的操作来
支持这一扩展:
\begin{minted}[mathescape=true,
               %linenos,
               numbersep=5pt,
               autogobble,
               frame=lines,
               fontsize=\footnotesize,
               bgcolor=bg,
               framesep=2mm]{Haskell}
fetchS    ::  S State
fetchS    =   \s -> (s, s)
\end{minted}
\noindent{}这会返回当前的计数。之后,我们扩充term数据类型:
\begin{minted}[mathescape=true,
               %linenos,
               numbersep=5pt,
               autogobble,
               frame=lines,
               fontsize=\footnotesize,
               bgcolor=bg,
               framesep=2mm]{Haskell}
data  Term        =  ... | Count

interp Count e    =  fetchS `bindS` (\i -> unitS (Num i))
\end{minted}
\noindent{}对\texttt{Count}求值会获取到至此为止的所有执行次数,并将其作为term的值返回。举个例子,如果
我们将\texttt{test}应用到\texttt{(Add (Add (Con 1) (Con 2)) Count)}上,会返回\texttt{"Value: 4; Count: 2"},
因为在\texttt{Count}被求值之前有一次相加操作。

\subsection{变体四:输出}\label{sec:output}
下面我们修改解释器以进行输出操作。虽然state monad看起来是一个自然的选择,但它对此有一点不足:将输出累积到
最终的状态意味着在计算结束之前不会有任何输出被打印,这不是我们想要的。下面的设计会在输出动作发生时就进行打印,
它依赖于惰性求值(lazy evaluation)。
\begin{minted}[mathescape=true,
               %linenos,
               numbersep=5pt,
               autogobble,
               frame=lines,
               fontsize=\footnotesize,
               bgcolor=bg,
               framesep=2mm]{Haskell}
type  O a       =  (String, a)

unitO a         =  ("", a)
m `bindO` k     =  let (r, a) = m
                       (s, b) = k a
                   in  (r ++ s, b)
showO (s, a)    =  "Output: " ++ s ++ " Value: " ++ showval a
\end{minted}
\noindent{}每个值都与计算其值时产生的输出成组。函数\texttt{unitO}返回给定的值而不产生任何输出。函数\texttt{bindO}执行
一个应用,并将其参数产生的输出与对应的应用产生的输出连接起来。函数\texttt{showO}用于打印输出和值。上述的这些函数都传播
了输出,但是并没有生成输出,生成输出的工作是下面的函数要做的事情:
\begin{minted}[mathescape=true,
               %linenos,
               numbersep=5pt,
               autogobble,
               frame=lines,
               fontsize=\footnotesize,
               bgcolor=bg,
               framesep=2mm]{Haskell}
outO      :: Value -> O ()
outO a    =  (showval a ++ "; ", ())
\end{minted}

\indent{}如此一来,语言又可以被扩展以支持输出操作。具体来说,将monad \texttt{M}用monad \texttt{O}替换,并
增加一个新的term和相应的clause:
\begin{minted}[mathescape=true,
               %linenos,
               numbersep=5pt,
               autogobble,
               frame=lines,
               fontsize=\footnotesize,
               bgcolor=bg,
               framesep=2mm]{Haskell}
data  Term          =  ... | Out Term

interp (Out u) e    =  interp u e `bindO` (\a  ->
                       outO a     `bindO` (\() ->
                       unitO a))
\end{minted}
\noindent{}对\texttt{(Out u)}求值会将\texttt{u}的值送往输出,并将其作为term的值返回。举个例子,如果
我们将\texttt{test}应用到\texttt{(Add (Out (con 41)) (Out (Con 1)))}上,会返回\texttt{"Output: 41; Value: 42"}。

\vspace{-0.2em}
\subsection{变体五:非确定性选择(Non-deterministic choice)}
\noindent{}现在修改解释器以应对非确定性语言会返回一列可能的结果的情形。所谓的 list monad 定义如下:
\begin{minted}[mathescape=true,
               %linenos,
               numbersep=5pt,
               autogobble,
               frame=lines,
               fontsize=\footnotesize,
               bgcolor=bg,
               framesep=2mm]{Haskell}
type  L a      =  [a]

unitL a        =  [a]
m `bindL` k    =  [b | a <- m, b <- k a]

zeroL          =  []
l `plusL` m    =  l ++ m

showL m        =  showlist [showval a | a <- m]
\end{minted}
\noindent{}上述片段使用了列表推导式(list comprehension)的记法进行描述。函数\texttt{showlist}取一个字符串列表,
并利用合适的符号从其产生一个字符串。

\indent{}我们的解释型语言又可以被扩充了,这次增加了两个新的结构。我们同样做代换,将monad \texttt{M} 替换为monad \texttt{L},
并增加了两个新的term和相应的clause。
\begin{minted}[mathescape=true,
               %linenos,
               numbersep=5pt,
               autogobble,
               frame=lines,
               fontsize=\footnotesize,
               bgcolor=bg,
               framesep=2mm]{Haskell}
data  Term            =  ... | Fail | Amb Term Term

interp Fail e         =  zeroL
interp (Amb u v) e    =  interp u e `plusL` interp v e
\end{minted}
\noindent{}对\texttt{Fail}求值不会返回值,对\texttt{(Amb u v)}求值会返回所有\texttt{u}和\texttt{v}返回的值。

\indent{}举个例子,如果我们将 \texttt{test} 应用到
\begin{displaymath}
\texttt{(App (Lam "x" (Add (Var "x") (Var "x"))) (Amb (Con 1) (Con 2)))}
\end{displaymath}
上,会返回\texttt{"[2,4]"}。如我们所见,在不纯的语言中想要做出用于支持这样效用的更改是很困难的。

\subsection{变体六:反向状态(Backwards state)}
现在回到~\ref{sec:variation_state}~中的state例子中,惰性求值使一种奇异的变体成为可能:状态可以被反向传播。
这正是这一小节要说明的。所有需要做的事情实际上只是修改一下\texttt{bindS}的定义:
\begin{minted}[mathescape=true,
               %linenos,
               numbersep=5pt,
               autogobble,
               frame=lines,
               fontsize=\footnotesize,
               bgcolor=bg,
               framesep=2mm]{Haskell}
m `bindS` k    =  \s2 -> let (a, s0) = m s1
                             (b, s1) = k a s2
                         in  (b, s0)
\end{minted}
\noindent{}它取最终状态作为输入,并返回初始状态作为输出。像之前一样,值\texttt{a}被\texttt{m}产生出来,
然后被送入\texttt{k}中。但现在是初始状态被送入\texttt{k},中间状态从\texttt{k}到\texttt{m},最终状态
由\texttt{m}返回。\texttt{let}表达式中的两个clause是互递归的,所以这只能在惰性下工作。

\indent{}定义于~\ref{sec:variation_state}~中的\texttt{Count} term现在返回的是将要在它的求值和执行末端之间
进行的步数。像之前一样,若我们将\texttt{test}应用到\texttt{(Add (Add (Con 1) (Con 2)) Count)}上,同样会返回
\texttt{"Value: 4; Count: 2"},但内部的过程不再一样了:在\texttt{Count}被求值处之后有一次相加发生。一个
不可解的互依赖(被称为“黑洞”)会出现在这样一种不幸的情况中,即尚未执行的步骤依赖于\texttt{Count}返回的值。
在这种情况下,解释器会失败并终止,或异常地终止。

\indent{}这个例子似乎表现得有些刻意而为,但这种monad在实践中确实是会出现的。John Hughes 和 Philip Wadler
在分析一个文本处理算法的时候发现了它,这个算法要从左至右且从右至左地传递信息。

\subsection{Call-by-name 解释器}
\indent{}在~\ref{sec:interp_monad}~开始时给出的解释器是call-by-value的,这很容易地通过函数的类型看出来。
其中,函数以类型\texttt{Value -> M Value} 被表示,所以说一个函数的参数是值,尽管函数应用的结果可以
是一个计算。而与之对应的call-by-name的解释器如下所示(仅给出存在不同之处)。
\begin{minted}[mathescape=true,
               %linenos,
               numbersep=5pt,
               autogobble,
               frame=lines,
               fontsize=\footnotesize,
               bgcolor=bg,
               framesep=2mm]{Haskell}
data  Value             =  Wrong
                        |  Num Int
                        |  Fun (M Value -> M Value)

type  Environment       =  [(Name, M Value)]

interp                  :: Term -> Environment -> M Value
interp (Var x) e        =  lookup x e
interp (Con i) e        =  unitM (Num i)
interp (Add u v) e      =  interp u e `bindM` (\a ->
                           interp v e `bindM` (\b ->
                           add a b))
interp (Lam x v) e      =  unitM (Fun (\m -> interp v ((x,m):e)))
interp (App t u) e      =  interp t e `bindM` (\f -> apply f (interp u e))

lookup                  :: Name -> Environment -> M Value
lookup x []             =  unitM Wrong
lookup x ((y,n):e)      =  if x==y then n else lookup x e

apply                   :: Value -> M Value -> M Value
apply (Fun h) m         =  h m
apply f m               =  unitM Wrong
\end{minted}
\noindent{}现在,用于表示函数的类型是\texttt{M Value -> M Value},所以现在函数的参数是一个计算。相似的,环境
也变了,它不再包含值,而是包含计算。用于解释常量和加法的代码不变。用于变量和lambda抽象的代码看似没有变,但
实际上发生了微妙的变化:之前,变量被绑定到值,而现在他们被绑定到计算上(因而\texttt{lookup}发生了一点改变)。
用于函数应用的代码确实改变了:现在是函数被求值,但参数并不。

\indent{}如果修改~\ref{sec:variation_state}~中的执行计数,每次参数被求值时,其开销会被计数。因而对
\texttt{test term0}求值的话,现在会返回字符串\texttt{"Value 42; Count: 4"},因为将\texttt{10}加到\texttt{11}上
的开销会被计数两次。

\indent{}Monadic风格在这里体现出来的好处是,类型使得效用发生之处十分清晰。因而,我们可以通过审视类型来区分
call-by-value和call-by-name。若用impure的特性,相关行为的线索是很模糊的。

\subsection{Monad 定律}\label{sec:monad_law}
\noindent{}若要\texttt{(M,unitM,bindM)}是monad,必须满足下面的定律:
\begin{minted}[mathescape=true,
               %linenos,
               numbersep=5pt,
               autogobble,
               %frame=lines,
               escapeinside = ||,
               fontsize=\normalsize,
               %bgcolor=bg,
               framesep=2mm]{Haskell}
|$\textrm{Left unit: }$|     (unitM a) `bindM` k  =  k a

|$\textrm{Right unit: }$|    m `bindM` unitM  =  m

|$\textrm{Associative:}\;\,\,$|        m `bindM` (\a -> (k a) `bindM` (\b -> h b))
              =   (m `bindM` (\a -> k a)) `bindM` (\b -> h b)
\end{minted}
\noindent{}这些定律保证了monadic的组合性。它是可结合的,并有左单位元和右单位元。

\indent{}为了说明这些定律的用途,考虑证明\texttt{(Add t (Add u v))}和\texttt{(Add (Add t u) v)}会返回相同的结果
这一命题。首先对左侧进行化简:
\begin{minted}[mathescape=true,
               %linenos,
               numbersep=5pt,
               autogobble,
               %frame=lines,
               escapeinside = ||,
               fontsize=\footnotesize,
               %bgcolor=bg,
               framesep=2mm]{Haskell}
   interp (Add t (Add u v)) e
=  interp t e            `bindM` (\a ->
   interp (Add u v) e    `bindM` (\y ->
   add a y))
=  interp t e            `bindM` (\a ->
   (interp u e           `bindM` (\b ->
    interp v e           `bindM` (\c ->
    add b c)))           `bindM` (\y ->
   add a y))
=  interp t e            `bindM` (\a ->
   interp u e            `bindM` (\b ->
   interp v e            `bindM` (\c ->
   add b c               `bindM` (\y ->
   add a y))))
\end{minted}
\noindent{}上述前两步都是简单的展开,而第三步用到了结合律。相似的,右侧也可被化简为:
\begin{minted}[mathescape=true,
               %linenos,
               numbersep=5pt,
               autogobble,
               %frame=lines,
               escapeinside = ||,
               fontsize=\footnotesize,
               %bgcolor=bg,
               framesep=2mm]{Haskell}
   interp (Add (Add t u) v) e
=  interp t e            `bindM` (\a ->
   interp u e            `bindM` (\b ->
   interp v e            `bindM` (\c ->
   add a b               `bindM` (\x ->
   add x c))))
\end{minted}
至此,还有待证明:
\begin{minted}[mathescape=true,
               %linenos,
               numbersep=5pt,
               autogobble,
               %frame=lines,
               escapeinside = ||,
               fontsize=\normalsize,
               %bgcolor=bg,
               framesep=2mm]{Haskell}
add a b `bindM` (\x -> add x c)  =  add b c `bindM` (\y -> add y a)
\end{minted}

\indent{}我们可以通过分例讨论来证明它。如果\texttt{a},\texttt{b}和\texttt{c}分别具有
形式\texttt{Num i},\texttt{Num j}和\texttt{Num k},那么左右两侧结果均可以表示
为\texttt{unitM (i+j+k)},这一点可由left unit定律和加法的结合性得到。其余情况的结果
是\texttt{Wrong},也由left unit定律得到。

\indent{}再来看一个例子。对于每个monad,我们可以定义下述操作:
\begin{minted}[mathescape=true,
               %linenos,
               numbersep=5pt,
               autogobble,
               frame=lines,
               fontsize=\footnotesize,
               bgcolor=bg,
               framesep=2mm]{Haskell}
mapM        :: (a -> b) -> (M a -> M b)
mapM f m    =  m `bindM` (\a -> unitM (f a))

joinM       :: M (M a) -> M a
joinM z     =  z `bindM` (\m -> m)
\end{minted}
\noindent{}对于list monad,\texttt{mapM}就是我们所熟知的映射(map)函数,且\texttt{joinM}是列表的连接。
我们约定\texttt{id}为恒等函数,符号\texttt{(.)}用作函数组合:\texttt{(f.g) x = f (g x)},则有下述定律:
\begin{minted}[mathescape=true,
               %linenos,
               numbersep=5pt,
               autogobble,
               frame=lines,
               fontsize=\normalsize,
               %bgcolor=bg,
               framesep=2mm]{text}
mapM id     =  id                               joinM . unitM       =  id
mapM (f.g)  =  mapM f . mapM g                  joinM . mapM unitM  =  id
                                                joinM . mapM joinM  =  joinM . joinM
mapM f . unitM  =  unitM . f
mapM f . joinM  =  joinM . mapM (mapM f)        m `bindM` k  =  joinM (mapM k m)
\end{minted}

\indent{}很多时候,monad并不是利用\texttt{unitM}和\texttt{bindM}给出其定义,而是用\texttt{unitM},\texttt{joinM}
和\texttt{mapM}来定义的。这样一来,前述的三条定律就被上面八条定律中的前七条所取代。如果再用第八条定律定义
\texttt{bindM},那么前述三条定律亦随之确立。因此,这两种定义是等价的。

\indent{}最后让我们以list monad为例,针对列表推导式的一些表达来看看这两种定义:
\begin{minted}[mathescape=true,
               %linenos,
               numbersep=5pt,
               autogobble,
               frame=lines,
               fontsize=\footnotesize,
               bgcolor=bg,
               framesep=2mm]{Haskell}
[t]                       =  unitM t                 =  unitM t

[t | x <- u]              =  mapM (\x -> t) u        =  u `bindM` (\x -> unitM t)

[t | x <- u, y <- v]      =  joinM (mapM (\x -> mapM (\y -> t) v) u)
                          =  u `bindM` (\x -> v `bindM` (\y -> unitM t))
\end{minted}

\vspace{-0.8em}
\section{Monad 续.}
\indent{}这一章的目的是对比monadic风格与延续传递风格(CPS)。
延续传递风格建立之始的目的是用于 指称语义(denotational semantics)。它为一个程序的执行序提供了
精细控制,并且作为一种编译器中间语言而流行。本文主要关注点在于CPS的模块化。

\subsection{CPS 解释器}
\noindent{}延续的monad(即continuation monad)定义如下:
\begin{minted}[mathescape=true,
               %linenos,
               numbersep=5pt,
               autogobble,
               frame=lines,
               fontsize=\footnotesize,
               bgcolor=bg,
               framesep=2mm]{Haskell}
type  K a      =  (a -> Answer) -> Answer

unitK a        =  \c -> c a
m `bindK` k    =  \c -> m (\a -> k a c)
\end{minted}

\indent{}在CPS中,一个(类型\texttt{a})的值被表示为一个函数,
这个函数取一个(类型为\texttt{a -> Answer}的)延续\texttt{c},并将延续应用到那个值上,给出
(类型为\texttt{Answer}的)最终结果\texttt{c a}。因此,\texttt{unitK a}给出了\texttt{a}的CPS表达。
若有\texttt{m :: K a}与\texttt{k :: a -> K b},那么\texttt{m `bindK` k}的行为是:将\texttt{c}绑定到
当前的延续上;对\texttt{m}求值;将求值结果绑定到\texttt{a}上;在延续\texttt{c}下将\texttt{k}应用到\texttt{a}。

\indent{}与之前我们对解释器进行修改的手法类似,将解释器中的monad \texttt{M}用monad \texttt{K}替换掉,然后做一下
简化,即可得到CPS下的解释器。

\begin{minted}[mathescape=true,
               %linenos,
               numbersep=5pt,
               autogobble,
               frame=lines,
               fontsize=\footnotesize,
               bgcolor=bg,
               framesep=2mm]{Haskell}
interp                :: Term -> Environment -> (Value -> Answer) -> Answer
interp (Var x) e      =  \c -> lookup x e c
interp (Con i) e      =  \c -> c (Num i)
interp (Add u v) e    =  \c -> interp u e (\a ->
                               interp v e (\b ->
                               add a b c))
interp (Lam x v) e    =  \c -> c (Fun (\a -> interp v ((x,a):e)))
interp (App t u) e    =  \c -> interp t e (\f ->
                               interp u e (\a ->
                               apply f a c))
\end{minted}
\noindent{}函数\texttt{lookup},\texttt{add}和\texttt{apply}现在都接受延续。定义\texttt{Add}的那一行可以读作:令\texttt{c}为
当前延续;对\texttt{u}求值;将\texttt{a}绑定到其结果上;对\texttt{v}求值;将\texttt{b}绑定到其结果上;在延续\texttt{c}下
将\texttt{a}和\texttt{b}加在一起。

\indent{}这个读法和~\ref{sec:what_is_monad}~中的monadic版本非常接近:CPS的版本可以由monadic的版本导出,只要将
\texttt{`bindM`}丢掉,并在前面和后面加些东西,即捕获和传递延续\texttt{c}。\texttt{`bindM`}的第二个参数具有类型
\texttt{a -> (b -> Answer) -> Answer},这是\texttt{k}的范围。延续具有类型\texttt{b -> Answer},这是\texttt{c}的
范围。\texttt{k}与\texttt{c}角色相似,它们二者都作为位于不同层次的延续。

\indent{}类型\texttt{Answer}可以是任何足以表示一个计算最终结果的类型。一个选择是让其为一个值。
\begin{minted}[mathescape=true,
               %linenos,
               numbersep=5pt,
               autogobble,
               %frame=lines,
               fontsize=\normalsize,
               %bgcolor=bg,
               framesep=2mm]{Haskell}
type  Answer    =  Value
\end{minted}
\noindent{}这决定了\texttt{showK}的定义。
\begin{minted}[mathescape=true,
               %linenos,
               numbersep=5pt,
               autogobble,
               %frame=lines,
               fontsize=\normalsize,
               %bgcolor=bg,
               framesep=2mm]{Haskell}
showK m         = showval (m id)
\end{minted}
\noindent{}这里\texttt{m :: K Value}将恒等函数\texttt{id}作为延续接受,然后得到的结果\texttt{Value}被转换为字符串形式。
对\texttt{test term0}求值会得到结果 \texttt{"42"},和前述结果相同。

\subsection{Call with current continuation(call/cc)}
\indent{}我们已经将解释器转换成了CPS的,现在我们可以进一步直接加上call-with-current-continuation(call/cc)操作。
这种操作可见于Scheme和New Jersey的Standard ML。
\begin{minted}[mathescape=true,
               %linenos,
               numbersep=5pt,
               autogobble,
               frame=lines,
               fontsize=\footnotesize,
               bgcolor=bg,
               framesep=2mm]{Haskell}
callccK        :: ((a -> K b) -> K a) -> K a
callccK h      =  \c -> let k a = \d -> c a
                        in  h k c
\end{minted}
\noindent{}这个操作捕获当前延续,并将其传入当前表达式。函数\texttt{callccK}的参数是一个函数\texttt{h},这个函数\texttt{h}
以类型为\texttt{a -> K b}的函数\texttt{k}为参数。如果\texttt{k}以参数\texttt{a}进行调用,它会忽略它的延续\texttt{d},然后
将\texttt{a}传入已捕获到的延续\texttt{c}中。

\indent{}现在,我们将call/cc加入到解释器语言中来,我们为当前的语言添加一个合适的term,并为解释器增加一个新的case。
\begin{minted}[mathescape=true,
               %linenos,
               numbersep=5pt,
               autogobble,
               frame=lines,
               fontsize=\footnotesize,
               bgcolor=bg,
               framesep=2mm]{Haskell}
data  Term               =  ... | Callcc Name Term

interp (Callcc x v) e    =  callccK (\k -> interp v ((x, Fun k):e))
\end{minted}
\noindent{}\texttt{callccK}被用来捕获当前的延续\texttt{k},并且\texttt{x}被绑定到一个行为与\texttt{k}一样的函数上,并被加入环境。
在这样的环境下,对\texttt{v}进行求值。

\indent{}举个例子来看看它的工作方式。将\texttt{test}应用到
\begin{displaymath}
\textrm{\texttt{(Add (Con 1) (Callcc "k" (Add (Con 2) (App (Var "k") (Con 4)))))}}
\end{displaymath}
上,会得到结果 \texttt{"5"}。

\subsection{Monads与CPS}
\indent{}我们已经看到了,通过选择合适的monad,monad解释器成为了一个CPS解释器。反过来也是成立的:通过选择合适的
解答空间(space of answers),一个CPS解释器可以表现为一个monad解释器。
通用(general)技巧如下。要在CPS下达到与monad \texttt{M}相同的效用,我们可以重定义answer类型,使其
包含\texttt{M}的一个application:
\begin{minted}[mathescape=true,
               %linenos,
               numbersep=5pt,
               autogobble,
               %frame=lines,
               fontsize=\normalsize,
               %bgcolor=bg,
               framesep=2mm]{Haskell}
type  Answer    =  M Value
\end{minted}
\noindent{}\texttt{showK}的定义也相应地被修改:
\begin{minted}[mathescape=true,
               %linenos,
               numbersep=5pt,
               autogobble,
               %frame=lines,
               fontsize=\normalsize,
               %bgcolor=bg,
               framesep=2mm]{Haskell}
showK n         =  showM (n unitM)
\end{minted}
\noindent{}这里\texttt{n :: K Value}接受\texttt{unitM :: Value -> M Value}作为延续。

\indent{}就像\texttt{unitM}将一个类型为\texttt{a}的值转换为类型\texttt{M a}一样,类型为\texttt{M a}的值可以被转换为
类型为\texttt{K a}的值,转换方式如下:
\begin{minted}[mathescape=true,
               %linenos,
               numbersep=5pt,
               autogobble,
               frame=lines,
               fontsize=\footnotesize,
               bgcolor=bg,
               framesep=2mm]{Haskell}
promoteK      :: M a -> K a
promoteK m    =  \c -> m `bindM` c
\end{minted}
\noindent{}因为有\texttt{m :: M a}和\texttt{c :: a -> M Value},则\texttt{m `bindM` c}的类型是\texttt{M Value},即为所需。

\indent{}举个例子,若要将错误消息纳入,我们取\texttt{M}为monad \texttt{E},然后做下述计算:
\begin{minted}[mathescape=true,
               %linenos,
               numbersep=5pt,
               autogobble,
               frame=lines,
               fontsize=\footnotesize,
               bgcolor=bg,
               framesep=2mm]{Haskell}
errorK      :: String -> (a -> E Value) -> E Value
errorK s    =  promoteK (errorE s)
            =  \c -> (errorE s) `bindE` c
            =  \c -> Error s    `bindE` c
            =  \c -> Error s
\end{minted}
\noindent{}通过分别应用\texttt{promoteK},\texttt{errorE}和\texttt{bindE}的定义来得出该等式。我们可以取最后一行作为
\texttt{errorK}的定义。就像我们所期望的那样,这会简单地扔掉延续并返回错误作为最终结果。

\indent{}上一节强调了monad支持模块化。举例来说,为了处理错误,仅需对monadic的解释器做一点修改:将monad \texttt{M}替换
为monad \texttt{E},并在合适的位置引入对\texttt{errorE}的调用。CPS也以相似的方式支持模块化。例如,为了处理错误,对CPS
解释器做出的修改也同样很简单:仅需改变\texttt{Answer}和\texttt{test}的定义,并在合适的位置引入对\texttt{errorK}的调用。

\indent{}执行计数(见~\ref{sec:variation_state}~节)与输出(见~\ref{sec:output}~节)可以用相似的方式被纳入CPS。对于
执行计数,我们可以取\texttt{Answer = S Value},并计算\texttt{tickS}的一个延续版本。
\begin{minted}[mathescape=true,
               %linenos,
               numbersep=5pt,
               autogobble,
               frame=lines,
               fontsize=\footnotesize,
               bgcolor=bg,
               framesep=2mm]{Haskell}
tickK    :: (() -> S Value) -> S Value
tickK    =  promoteK tickS
         =  \c -> tickS `bindS` c
         =  \c -> (\s -> ((), s+1)) `bindS` c
         =  \c -> \s -> c () (s+1)
\end{minted}
\noindent{}对于输出,我们可以取\texttt{Answer = O Value},并计算\texttt{outO}的一个延续版本。
\begin{minted}[mathescape=true,
               %linenos,
               numbersep=5pt,
               autogobble,
               frame=lines,
               fontsize=\footnotesize,
               bgcolor=bg,
               framesep=2mm]{Haskell}
outK      :: Value -> (Value -> O Value) -> O Value
outK a    =  promoteK (outO a)
          =  \c -> (outO a) `bindO` c
          =  \c -> (showval a ++ "; ", ()) `bindO` c
          =  \c -> let (s,b) = c () in (showval a ++ "; " ++ s, b)
\end{minted}
\noindent{}对于这两种情形,对CPS解释器版本的修改与像我们修改monadic的对应版本一样简单。

\subsection{Monads vs. CPS}
\indent{}上一节我们阐明了monad与CPS间的关系。一点令人好奇之处是,在monad与CPS之间是否存在真正的不同点。
对于monad,我们可以写出\texttt{m `bindM` (\textbackslash a -> k a)};
对于CPS,我们可以写出\texttt{(\textbackslash c -> m (\textbackslash a -> k a c))}。
具体要选哪一种,似乎只不过是偏好性的问题。

\indent{}实际上,的确是有不同之处的。我们所描述的每个monad类型都可能转变成抽象数据类型(abstract data type),并可
在某种程度上提供比CPS更为精细的控制。比如,我们已经看到,monad类型 \texttt{S a}的CPS类似物是下述类型:
\begin{minted}[mathescape=true,
               %linenos,
               numbersep=5pt,
               autogobble,
               %frame=lines,
               fontsize=\normalsize,
               %bgcolor=bg,
               framesep=2mm]{Haskell}
(a -> S Value) -> S Value.
\end{minted}
\noindent{}后一种类型包含了诸如下类的值:
\begin{minted}[mathescape=true,
               %linenos,
               numbersep=5pt,
               autogobble,
               %frame=lines,
               fontsize=\normalsize,
               %bgcolor=bg,
               framesep=2mm]{Haskell}
\c -> \s -> (Wrong, c)
\end{minted}
\noindent{}这提供了一种错误逃逸:它忽略了当前的延续,并总是返回\texttt{Wrong},而state monad \texttt{S}则无此便利。对于
monad,我们可以选择是否要提供逃逸能力,而CPS没办法做这种选择。

\indent{}或许在它们二者之间的一个更有意义的区别是看待问题视角的不同。Monad将注意力集中在需要什么抽象
操作,这些操作需要满足什么定律,如何将不同monad所表达的特性进行结合的问题上。


\section{体验Monads}
\noindent{}Haskell编译器的每个阶段都与monad相关:
\begin{itemize}
\item 类型推导阶段使用了一个monad,它带有一个error部分(与~\ref{sec:err_msg}~中的\texttt{E}类似),
一个position部分(与~\ref{sec:err_msg_pos}~中的\texttt{P}类似)和两个
state部分(与~\ref{sec:variation_state}~中的\texttt{S}类似)。对于两个state部分,其一用作name supply以产生独一无二
的新变量名。另一是current 代换,用于unification。
\item \textsl{Simplification}阶段使用了一个monad,它带有一个用作name supply 的state部分。
\item 代码生成阶段使用了一个带有三个state部分的monad:
    \begin{itemize}
        \item 一个list用来放到目前为止生成的代码
        \item 一个table用于将变量名与寻址模式进行关联
        \item 另一个table用于缓存执行期栈状态的已知内容
    \end{itemize}
\end{itemize}

\indent{}在每种情况下,monad的使用都极大地简化了薄记(bookkeeping)工作。如果类型推导器需要在每一步都明确地提及当前替
换,name supply和error信息是如何传播的,就会变得非常混乱。

\end{document}