\documentclass[12pt]{article}
\usepackage{array}
\usepackage{amsmath}
\usepackage{amssymb}
\usepackage{xfrac}
\usepackage{ntheorem}
\usepackage{algorithm}
\usepackage{algorithmic}
\usepackage{caption}
\usepackage{fontspec}
\usepackage{graphicx}
\usepackage{indentfirst}
\usepackage{enumitem}
\usepackage{minted}
\usepackage{mathtools}
\usepackage{pifont}
\usepackage{setspace}
\usepackage{subfigure}
\usepackage{tikz}
\usepackage{url}
\usepackage{tcolorbox}
\usepackage{xcolor}
\usepackage{xeCJK}

\usepackage[colorlinks=true]{hyperref}
\usepackage[margin=0.55in]{geometry}

% background color for minted
\definecolor{bg}{rgb}{0.95,0.95,0.95}

% CJK font
\setCJKmainfont{Source Han Serif CN}

% indent value
\setlength{\parindent}{2em}

% line spacing
\linespread{1.2}
\renewcommand{\thesection}{\Roman{section}}
\renewcommand{\thesubsection}{\thesection-\arabic{subsection}}

\title{How to Start Learning Computer Graphics Programming}
\author{Yiteng Zhang}

\begin{document}
\maketitle

\noindent{}\textbf{注:}本文的内容来自Eric Arneb\"ack的博客文章,主要涉及如何以合理的方式开始学习计算机图形学编程
的问题。由于觉得作者写得非常好,所以不免想要将其记录下来。这篇文章的重要价值在于它能够引导初学者,从而避免在
初学时踩一些不必要的坑。实际上,如果没有正确的指引,一些坑总会很容易踩上去,并耗费初学者大量的时间,这种事对于迫切想要
开始图形学编程的初学者而言其实是非常常见的。更可怕的是,这些不友好的坑往往会使一位满怀热情的初学者很快打退堂鼓,从而无缘
进阶的有趣内容。一位无基础的学习者如果一开始就去学习OpenGL或Direct3D,那么上面描述的事情就很有可能会发生,毕竟
无基础意味着不能正确地将图形学的学习和图形API的学习分开来看,这其实是很危险的。与之相对的,在有了一定的基础之后再进行
图形API的学习,可以说是事半功倍。针对上述议题,Eric Arneb\"ack提出了一些很好的建议,在此将其转译整理。

\subsection*{建议一:从光线追踪(Raytracing)和光栅化(Rasterization)开始}
\indent{}近年来,用于针对GPU硬件来编码的API不断涌现,比如Direct3D、OpenGL、Vulkan、Metal还有WebGL等。从这些
API开始学习的话,可能会是非常困难的,因为在使用它们时往往会涉及很多样板代码。实际上,这些图形API对初学者是
非常不友好的,从它们开始是很糟糕的选择。对于毫无图形学基础的学习者而言,理解如何去绘制一个三角形都是颇为艰难的。
当然了,我们也可以不从它们开始,而是选择像Unity或Unreal这样的游戏引擎。使用游戏引擎在这里的好处是,与图形API交流
所涉及的那些繁琐工作将由它们完成。尽管如此,对于毫无基础的学习者而言,需要学习的有关游戏引擎的那套东西也还是太多了,
应该将学习时间用在“刀刃”上。

\indent{}对初学者的建议是,最好先自己动手写一下raytracer或software rasterizer(建议都搞一搞)。简单来说,\textbf{光线
追踪渲染器(Raytracer)}是这样一种程序:通过从屏幕(\textbf{注:}这里应理解为camera screen)上的每个像素发出许多射线,并
进行大量的求交与物理光照计算,来确定渲染所得图像上每个像素的最终颜色,从而渲出三维场景。而\textbf{软件光栅
化渲染器(Software Rasterizer)}是这样渲染三维场景(多数时候是一堆三角形)的:对于每个我们想要绘制的三角形,去寻找
它覆盖了屏幕上的哪些像素,然后我们计算与每个像素对应的三角形上的点是如何与光交互的,从而得到像素的最终颜色。之所以我们
这里说“软件”光栅化渲染器,是因为它是相对于“硬件”(即GPU)而言的。这意味着我们是根据光栅化渲染的原理手动实现它,而
不依赖于GPU,所以我们的software rasterizer跑在CPU上。

\indent{}光栅化和光线追踪的原理实际上都挺简单的。所以对于初学者而言,动手实现它们要比理解现代图形API容易得多。并且,通过
实现它们,我们可以接触到很多图形学基础概念,比如相机和变换矩阵等。这些时间不必浪费在与现代图形API较劲上(相信这带来的挫败
感已经劝退了很多人)。

\indent{}在学习某种图形API之前先写一个软件光栅化渲染器是很有好处的。一个很大的好处在于,我们会发现在使用图形API时如果
某个地方出了问题,debug起来变得更容易了。这些图形API实际上只是提供了一个到基于GPU的光栅化渲染的接口,既然你了解
这些API在幕后是如何工作的,那么debug起来自然会容易很多。

\indent{}要写一个raytracer的话,读一读Peter Shirley的那三本小书\footnote{即Peter Shirley的\textit{Ray Tracing in One Weekend}
系列}总是很好的选择。若是打算写一个software rasterizer,可以看看Scratchapixel上的相关内容。

\subsection*{建议二:学点必要的数学知识}
\indent{}第二点建议是学习一些计算机图形学基础所需的数学知识。这点其实不用太过担心,开展图形学编程所需的数学概念和方法
其实出乎意料地少。作为图形学初学者,所需的数学工具一般只是一点有关线性代数的内容,大多数时候用到的数学概念列举如下:
\begin{itemize}
\item 点积(Dot Product)
\item 叉积(Cross Product)
\item 球坐标(Spherical Coordinates)
\item 变换矩阵(Transformation Matrix):多数时候都在处理$4\times 4$矩阵,所以别花时间研究大矩阵
\item 旋转/缩放/平移矩阵(Rotation/Scaling/Translation Matrix)
\item 齐次坐标(Homogeneous Coordinates)
\item 四元数(Quaternions)
\item 求交计算(Intersection calculations):大部分时候是对射线与球或者射线与平面或三角形的求交
\item 列主序(Column-major Order)与行主序(Row-major Order):处理起来有点绕,请确保完全理解
\item 如何用view matrix与perspective matrix建模相机:有点复杂,最好仔细深入学习一下
\end{itemize}
\noindent{}从初学者到有点基础的这个阶段,大多时候遇到的数学都不会超出上述范围。不过,一旦当你踏入“基于物理的渲染”
这一主题时,微积分的相关知识也是必须的,但对于初学而言暂时还用不着。

\subsection*{建议三:针对“绘制第一个三角形”时可能遇到的问题的一些Debug提示}
\indent{}一旦写过一个raytracer或rasterizer,在学习图形API时就会更有自信。对于图形API的学习而言,你的hello world程序
一般是在屏幕上绘制一个三角形。这听起来非常简单,但实际上它可能出乎意料地难,因为必须要写很多样板代码,而且图形学代码
debug起来对初学者来说也并不容易。如果在绘制第一个三角形时出了问题,只看到一个黑屏而没有三角形,可以尝试下面列出的debug
思路:
\begin{itemize}
\item 通常,问题出在projection matrix或view matrix上,它们挺容易出错的。在\textbf{顶点着色器(Vertex Shader)}中,每个
顶点被model matrix、view matrix和projection matrix依次应用,最后再做一下perspective divide(这一步一般是在幕后自动完成的,不
需要我们显式地做)。试着手动完成这个过程,好好检查一下用到的矩阵。如果你希望一个顶点是可见的,那么在做完perspective divide后它
应该处于标准化设备坐标系中。其坐标的$x$与$y$的值应该都位于$[-1,+1]\in \mathbb{R}$。如果使用的是OpenGL,那么$z$的取值应该也在
$[-1,+1]\in\mathbb{R}$中。但如果使用的是Direct3D,$z$的取值应该在$[0,+1]\in \mathbb{R}$中。坐标值具体在什么范围内取决于使用的
API的相应约定,虽然这本身无关紧要,但为了得到正确的结果还是要注意检查一下。如果坐标值不在期望的范围内,那么你本希望可见
的顶点实际上是不可见的,因为它们被剪裁掉了。
\item 有没有记得用合理的值将depth buffer清理?举例来说,如果你在使用Direct3D,用了深度比较函数\texttt{D3DCMP\_LESS},并接着把
depth buffer清理为0,那么任何东西都不会被绘制,因为顶点都通不过深度测试。总之,请确保完全理解\textbf{深度测试(Depth Test)},
并正确地配置它。
\item 确保矩阵被正确地传递给GPU。作为辅助,我们可以使用一些debug工具来验证它们是否被正确传递,比如RenderDoc。与此类似,请确保
所有顶点数据也被正确地传递给GPU。只传递一部分顶点数据是一个很常见的错误。
\item 另一个使人困扰的细节是\textbf{背面剔除(Back-face Culling)}。以OpenGL举例,背面三角形会被默认剔除掉。所以如果你搞了一个
背面三角形并打算去渲它,是不会看到关于它的渲染结果的,因为被剔除掉了。关于这一点,建议先暂时把背面剔除禁用,这样可以让事情变得
更清晰一些。
\item 检查图形API给出的所有错误代码,里面可能会包含有用的信息。如果你所使用的API可以访问某种调试层(debugging layer),比
如使用Vulkan这种API,那么最好将它开启。
\item 对于任何类型的图形学编程,都强烈建议学着去用一用相关的GPU调试工具,比如RenderDoc或者Nsight之类的。这些工具可以
给你提供与你的图形应用程序每一步对应的GPU状态信息的预览,方便你检查是否一切都处于正常工作的状态。
\end{itemize}

\subsection*{建议四:一些不错的练手项目}
\indent{}自己动手去实现一下各种图形学技术,可能是学好图形学编程的最好方法。关于这一点,下面给出了一些适合新手实践的项目:
\begin{itemize}
\item 使用球坐标系做一个球面网格(sphere mesh),然后把它渲出来
\item 实现漫反射(Diffuse)和镜面反射(Specular)着色器
\item 实现方向光(Directional Lights)、点光源(Point Lights)和聚光灯(Spot Lights)
\item 实现高度图(Heightmap)渲染
\item 实现某种简单网格格式(比如.obj格式)的解析器,然后将网格导入到项目中进行渲染,并尝试导入和渲染带纹理(textures)的网格
\item 实现一个简单的Minecraft风格的渲染器,这不是很难,而且很有学习价值
\item 使用立方体贴图(Cube Mapping)来渲染反射效果
\item 使用阴影贴图(Shadow Mapping)来渲染阴影效果
\item 实现视锥体剔除(View Frustum Culling),这是简单而实用的优化技术
\item 实现粒子系统(Particle system)的渲染
\item 学习如何实现伽马矫正(Gamma Correction)
\item 实现法线贴图(Normal Mapping)
\item 学习使用实例渲染(Instanced Rendering)来高效渲染大量网格
\item 使用网格蒙皮(Skinning)来做动画
\end{itemize}
\noindent{}还有一些更高级一点的技术:
\begin{itemize}
\item 实现各种后处理效果,比如使用高斯模糊做辉光(Bloom)、屏幕空间环境光遮蔽(SSAO)、抗锯齿/反走样(Anti-aliasing)、
快速近似抗锯齿(FXAA)等
\item 实现延迟着色(Deferred Shading),这是一种有效的多光源渲染技术
\end{itemize}

\indent{}还有很多有趣有用的技术,这里不再给出了。相信通过完成上面的实践,足以建立起对图形学基础和渲染基础
的基本了解,并已经在这个过程中接触到更多进阶主题,能够根据自身兴趣继续探索。

\end{document}